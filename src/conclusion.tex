\begin{conclusion}

% TODO: uvest, ze ta aplikace je sice naimplementovana, ale jeste ji neco malo chybi

  V práci jsem pro lepší pochopení uvedl čtenáře do problematiky, analyzoval již fungující i možná řešení
  a~provedl návrh backendu, ze kterého vzešlo řešení vytvořit webovou službu nezávislou na webové aplikaci.
  Službu se podařilo implementovat pomocí jazyka Python a úspěšně nasadit na virtuální privátní server.
  Důležitým výsledkem je také sepsání dokumentace a vytvoření automatizovaných testů pro~lepší odhalování chyb,
  jimiž mohu usnadnit práci nejen sobě, ale i všem dalším programátorům, kteří se budou na vývoji v budoucnu podílet. % Hlavním úspěchem je tak kompletní dokončení projektu.
  
  Teď, když je celá práce za mnou, mohu s odstupem prohlásit, že více času jsem se měl věnovat návrhové části.
  Systém měl být daleko více objektově orientovaný. Ve zdrojových kódech dochází k nelogickému umístění částí kódu,
  které se měly nacházet na vhodnějších místech, například v třídních metodách a ne ve funkcích apod.
  Kromě toho je určitě prostor pro zlepšení webové služby po stránce bezpečnosti, především použití protokolu HTTPS.
  
  Hlavním cílem práce bylo navrhnout a implementovat backend pro webovou aplikaci. I přesto, že moje původní představa
  týkající se malého rozsahu práce byla poměrně naivní, úspěšně jsem dokončil hlavní i~dílčí cíle. Webová aplikace kolegy Jaroslava Veselého
  úspěšně využívá moje řešení a na základě naší debaty máme další nápady, jak aplikaci rozšířit. Na základě analýzy jsme také zjistili,
  že podobná aplikace pro jednoduchou správu turnajů v Ultimate Frisbee zde chybí a úspěšným dokončením této práce jsme vyplnili díru na~trhu.  
\end{conclusion}