\chapter{Implementace}

\section{Výběr technologií}

\indent

V době, kdy existuje obrovské množství programovacích jazyků, jejich ekosystémů a dalších frameworků je důležité se umět nespálit. 
Vybrat špatné technologie totiž může znamenat prodloužení vývoje, zvýšení celkových nákladů nebo dokonce ukončení projektu.
Proto je nutnost tuto část dostatečně analyzovat a svědomitě zvážit, aby nedošlo k omylu, který by zabraňoval úspěšné dokončení.
Pro adekvátní výběr tak bylo potřeba specifikovat příslušná kritéria, které by jasně porovnala všechny zvažované technologie:

\subsubsection*{Podpora webových služeb}
Při tvorbě webové aplikace nemůže být ani řeč o technologii, která by pro použití na webu nebyla vhodná. Technologie musí
umět pracovat s mapováním URL a HTTP metod na zdroje a pracovat s běžnými formáty dat v internetové komunikaci (např. JSON).

\subsubsection*{Dokumentace a uživatelská podpora}
Pro pochopení technologie a její správné používání je často potřeba znát detaily, které se uživatel dozví z dokumentace.
Ta dokáže často výrazně zkrátit učící křivku a programátorovi tak šetří čas. Pokud dokumentace nestačí, může často její
nedostatky zachránit dostatečně široká uživatelská základna. Velká uživatelská podpora navíc snižuje riziko zestárnutí zvolené technologie.

\subsubsection*{Znalost technologie}
Pokud již programátor některou z technologií zná, může mu tato znalost ušetřit mnoho času při vývoji. Nemusí totiž novou technologii
nijak složitě zavádět nebo se učit její použití. Vhodným výběrem se také může vyvarovat situacím, kdy až v průběhu implementace narazí
na neřešitelné problémy. Zavrhnout neznámé technologie ale může znamenat přijít o možnost používat nástroje, které jsou problémům šité na míru. 

\subsubsection*{Výkon}
I když aplikace nepočítá s vetším provozem čítající stovky požadavků za sekundu, není důvod používat technologie, které jsou pomalé
nebo spotřebovávají příliš moc prostředků.

\section{Zvolené technologie}

\subsection{Programovací jazyk Python}

\indent

Python je víceúčelový skriptovací jazyk navrhnut v roce 1991 \cite{python-year} Guido van Rossumem.
Je vyvíjen jako open source projekt a je bezplatně dostupný pro většinu dostupných platforem (Unix, Windows, Mac OS).
V distribucích Linuxu je často součástí základní instalace. Díky velkému množství knihovních modulů z různých oblastí
má velice široké možnosti uplatnění. Běžně jej používají ve firmách jako Google, Dropbox, YouTube, Red Hat, Cisco,
Facebook nebo Microsoft \cite{python-companies}.

\medskip

Protože jde o dynamický interpretovaný jazyk, jeho zdrojový kód není nutné překládat překladačem do strojového kódu.
Je to zárovéň hybridní jazyk, protože umožňuje používat různá programovací paradigmata, včetně objektově orientovaného,
imperativního, procedurálního nebo v omezené míře i funkcionálního. I když je Python mnohokrát označován za skriptovací jazyk,
jeho návrh umožňuje psaní rozsáhlých a plnohodnotných aplikací včetně GUI. Podporuje také dynamickou kontrolu datových typů.

\medskip

Je to jazyk, který se velmi snadno učí a bývá považován za jeden z nejvhodnějších programovacích jazyků pro začátečníky.
Pomáhá tomu jeho jednoduchá syntaxe a čistota kódu. Na rozdíl od jiných jazyků bývá jeho zdrojový kód často krátký a dobře čitelný.
Je tak vhodný pro výuku i využití v praxi. Podle PYPL indexu\footnote{PopularitY of Programming Language Index je vytvořen analyzováním toho,
jak často jsou tutoriály jednotlivých programovacích jazyků hledány na Googlu.} je Python celosvětově nejrychleji roustoucí programovací jazyk
v~oblíbenosti za posledních pět let \cite{python-pypl}.
V dubnovém žebříčku roku 2016 drží druhé místo mezi jazyky Java a PHP. Mnoho dalších žebříčků pak uvádí Python vždy v prvních pěti místech.

\medskip

Po stránce výkonu je na tom Python relativně dobře, protože mnoho na~výkon náročných knihoven je implementováno v~jazyce C.
V porovnání s ostatními interpretovanými jazyky je na tom samotný Python taky dobře.
Fakt se kterým ale musíme počítat je ten, že dynamicky interpretované jazyky jsou obecně pomalejší, než kompilované jazyky.

\subsection{Webový framework Falcon}

% TODO: dovysvetlit, co je to UWSGI, mozna dat pred Falcon 

\indent

Falcon je minimalistický webový framework pro vývoj aplikačních backendů a jejich API s otevřeným zdrojovým kódem,
populární pro svoji neuvěřitelnou rychlost. Falcon ctí architektonický styl REST, což znamená, že mapuje použité
zdroje a jejich metody do HTTP protokolu.

\medskip

Falcon disponuje minimálním množstvím závislostí na jiných knihovnách a díky jeho flexibilitě je ho možné
používat ve většině verzích Pythonu\footnote{Vývoj v Pythonu poznamenalo v roce 2008 vydání nové verze Python 3.0,
která je částečně zpětně nekompatibilní.}. Kromě toho umí pracovat s WSGI. Nevýhodou Falconu je menší uživatelská
základna na rozdíl od konkurenčního Flasku, jenž patří mezi nejpopulárnější webové frameworky pro Python.

\begin{table}[htb]
 \centering
 \begin{tabular}{|l||c|c|c|c|}\hline
 \bfseries \bfseries framework & \bfseries req/sec & \bfseries $\mu$s/req & \bfseries výkon \\[2mm]
 \hline
 Falcon (0.3.0) & 21,858 & 46 & 8x \\
 \hline
 Bottle (0.12.8) & 12,583 & 79 & 4x \\
 \hline
 Werkzeug (0.10.4) & 4,708 & 212 & 2x \\
 \hline
 Pecan (0.8.3) & 3,442 & 291 & 1x \\
 \hline
 Flask (0.10.1) & 2,837 & 352 & 1x \\
 \hline
 \end{tabular}
 \caption{Výkonnostní test několika podobných webových frameworků pro~CPython~2.7.9 \cite{falcon-benchmarks}. Jde o~implementaci jazyka Python, kterou používá Catcher.}
\end{table}

\subsection{Databáze MySQL}

\indent

Relační datábáze, se kterou probíhá komunikace pomocí jazyka SQL, jak už název napovídá.
Díky svému výkonu, snadné použitelnosti (lze ji nainstalovat na Linux, Windows, OS X a další) a faktu,
že jde o volně šiřitelný software (je k dispozici i pod komerční licencí),
je MySQL velmi populární. Je součástí velmi oblíbené kombinace
základního softwaru na serverech známou pod zkratkou LAMP\footnote{Linux, Apache, MySQL, PHP}.

\medskip

Pro správu databáze se používá zpravidla příkazový řádek nebo lze separátně stáhnout
a nainstalovat nástroj zvaný MySQL Workbench. Ten byl použit i v této práci při návrhu databázového modelu.
Od svého vzniku v roce 1995 [ZDROJ] se od MySQL odpojilo několik alternativních větví (tzv. fork).
Mezi nejznámější případy patří MariaDB [ZDROJ] a Percona [ZDROJ].

\subsection{Peewee ORM}

\indent

Peewee je implementace ORM (co je to ORM a co se snazi, prestat psat sql dotazy, nevyhody?) pro jazyk Python.
Obsahuje podporu pro databáze SQLite, PostgreSQL a MySQL. Je otevřeným softwarem a jeho zdrojový kód
je dostupný na GitHubu.

\subsection{Verzovací systém Git}

\textit{Povídání o Gitu a GitHubu.}

\subsection{Webový server Nginx}

\indent

Nginx je webový server a reverzní proxy\footnote{Reverzní proxy se používá pro zvýšení výkonu webového serveru.
Rozděluje vstupující provoz na více serverů, například vyvažuje zátěž serverů zapojených v clusteru.}, který pracuje s běžnými protokoly.
První oficiální verze se objevila v roce 2004 \cite{nginx-changes}, kterou vyvynul Ruský softwarový inženýr Igor Sysoev.
Zaměřuje se na vysoký výkon a nízké nároky na pamět. Je používán velkými firmami
díky propracované možnosti rozložení zátěže. Z velkých firem je to například Netflix, Wordpress.com,
GitHub, Dropbox nebo Seznam.cz.

\medskip

Dnes je již s 25 \% druhým nejpoužívanějším webovým serverem v prvním milionu nejzatíženějších webů na světě.
Napříč celým webem je pak na třetím místě s přibližně 16 \%. I když je Apache HTTP Server stále jednička, 
jejich rozdíl se neustále zmenšuje \cite{nginx-statistic}. Nginx je open source.

% TODO: IDEA: Moznost prilozit obrazek s grafem nebo tabulkou.

\section{Bezpečnost}

\textit{Jak probíhá komunikace, jak se uživatelé autentizují a autorizují.}

\section{Další}

\textit{Popíšu vybrané části systému. Například jak aplikace komunikuje s databází, nebo jak se počítají SOTG, postupy ze skupin apod. Nemám ještě dovymšlené.}