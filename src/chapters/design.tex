\chapter{Návrh}

\textit{V této kapitole se podíváme na návrh aplikace, na jednotlivé RESTové zdroje,
adresářovou strukturu. Ukážeme konceptuální a databázový model a trochu si o nich něco řekneme.
Možná provedeme diskuzi smyček.}
%\section{Návrh tříd a databáze}
% TODO: Takovy ty kecy teoreticky. Jak ma vypadat diagram trid, databazovy model apod.


\textit{Zde bude další pokračování o RESTu. Povídání o zdroji, požadavcích na RESTful rozhraní, formátech zpráv (JSON).}

% TODO: tohle vsechno ostatni presunout do navrhu, abychom vedeli, jak spravne navrhnout restful prostredi

\paragraph{Požadavky na RESTové rozhraní}

Aby se mohlo nějaké aplikační rozhraní požadovat za RESTové (tzv. RESTful),
musí splňovat několik základních předpokladů definované Fieldingem [ZDROJ].
Nyní si uvedeme ty nejdůležitější:

\begin{description}
    \item[Architektura klient-server] \hfill \\
    Rozděluje zodpovědnost mezi různé části. Klientské aplikace se nemusí starat o správu dat a server o jejich prezentaci.
    Z toho vyplývá možnost vývíjet obě kompomenty (klient a server) nezávisle na sobě.
    \item[Bezstavovost] \hfill \\
    Aplikační stav je udržován na straně klienta. Všechny požadavky tak obsahují pouze informace nutné k jeho zpracování.  
    \item[Jednotné rozhraní] \hfill \\
    Lze získat reprezentaci nebo manipulovat s libovolným zdrojem pomocí unikátního identifikátoru.
    Z reprezentace zdroje musí být klient schopen s daným zdrojem manipulovat. Zprávy by měly být dostatečně popisné.
\end{description}
 
\paragraph{Zdroj}

Architektonický styl REST je postaven na základním prvku - zdroji. Jde o logický objekt,
který lze vyjádřit smysluplnou reprezentací, a s kterém lze manipulovat pomocí zpřístupněných metod.
% Zdroj má unikátní identifikátor u svoji URL


\paragraph{Formát zpráv}