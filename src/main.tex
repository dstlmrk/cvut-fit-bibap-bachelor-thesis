% options:
% thesis=B bachelor's thesis
% thesis=M master's thesis
% czech thesis in Czech language
% slovak thesis in Slovak language
% english thesis in English language
% hidelinks remove colour boxes around hyperlinks


\documentclass[thesis=B,czech]{FITthesis}[2012/06/26]

\usepackage[utf8]{inputenc} % LaTeX source encoded as UTF-8

\usepackage{graphicx} %graphics files inclusion
% \usepackage{amsmath} %advanced maths
% \usepackage{amssymb} %additional math symbols

\usepackage{dirtree} %directory tree visualisation


% --------------------------------------------------------------------
% Default fixed font does not support bold face
\DeclareFixedFont{\ttb}{T1}{txtt}{bx}{n}{9} % for bold
\DeclareFixedFont{\ttm}{T1}{txtt}{m}{n}{9}  % for normal
% Custom colors
\usepackage{color}
\definecolor{deepblue}{rgb}{0,0,0.5}
\definecolor{deepred}{rgb}{0.6,0,0}
\definecolor{deepgreen}{rgb}{0,0.5,0}
\usepackage{listings}
\lstset{
     literate=%
         {á}{{\'a}}1
         {í}{{\'i}}1
         {é}{{\'e}}1
         {ý}{{\'y}}1
         {ú}{{\'u}}1
         {ó}{{\'o}}1
         {ě}{{\v{e}}}1
         {š}{{\v{s}}}1
         {č}{{\v{c}}}1
         {ř}{{\v{r}}}1
         {ž}{{\v{z}}}1
         {ď}{{\v{d}}}1
         {ť}{{\v{t}}}1
         {ň}{{\v{n}}}1                
         {ů}{{\r{u}}}1
         {Á}{{\'A}}1
         {Í}{{\'I}}1
         {É}{{\'E}}1
         {Ý}{{\'Y}}1
         {Ú}{{\'U}}1
         {Ó}{{\'O}}1
         {Ě}{{\v{E}}}1
         {Š}{{\v{S}}}1
         {Č}{{\v{C}}}1
         {Ř}{{\v{R}}}1
         {Ž}{{\v{Z}}}1
         {Ď}{{\v{D}}}1
         {Ť}{{\v{T}}}1
         {Ň}{{\v{N}}}1                
         {Ů}{{\r{U}}}1    
}
% Python style for highlighting
\newcommand\pythonstyle{\lstset{
language=Python,
basicstyle=\ttm,
otherkeywords={self},             % Add keywords here
keywordstyle=\ttb\color{deepblue},
emph={TestClubs,__init__,setUp,tearDown,testGet,testPost},          % Custom highlighting
emphstyle=\ttb\color{deepred},    % Custom highlighting style
stringstyle=\color{deepgreen},
% frame=tb,                         % Any extra options here
showstringspaces=false            % 
}}
% Python environment
\lstnewenvironment{python}[1][]
{
\pythonstyle
\lstset{#1}
}
{}
% Python for inline
\newcommand\pythoninline[1]{{\pythonstyle\lstinline!#1!}}
% --------------------------------------------------------------------
    
% % list of acronyms
% \usepackage[acronym,nonumberlist,toc,numberedsection=autolabel]{glossaries}
% \iflanguage{czech}{\renewcommand*{\acronymname}{Seznam pou{\v z}it{\' y}ch zkratek}}{}
% \makeglossaries

\newcommand{\tg}{\mathop{\mathrm{tg}}} %cesky tangens
\newcommand{\cotg}{\mathop{\mathrm{cotg}}} %cesky cotangens

% % % % % % % % % % % % % % % % % % % % % % % % % % % % % % 
% ODTUD DAL VSE ZMENTE
% % % % % % % % % % % % % % % % % % % % % % % % % % % % % % 

\department{Katedra softwarového inženýrství}
\title{Systém pro skórování Ultimate~Frisbee~zápasů~-~backend}
\authorGN{Marek} %(křestní) jméno (jména) autora
\authorFN{Dostál} %příjmení autora
\authorWithDegrees{Marek Dostál} %jméno autora včetně současných akademických titulů
\supervisor{Ing. Jiří Hunka}
\acknowledgements{Doplňte, máte-li komu a za co děkovat. V~opačném případě úplně odstraňte tento příkaz.}
\abstractCS{V~několika větách shrňte obsah a přínos této práce v~češtině. Po přečtení abstraktu by se čtenář měl mít čtenář dost informací pro rozhodnutí, zda chce Vaši práci číst.}
\abstractEN{Sem doplňte ekvivalent abstraktu Vaší práce v~angličtině.}
\placeForDeclarationOfAuthenticity{V~Praze}
\declarationOfAuthenticityOption{4} %volba Prohlášení (číslo 1-6)
\keywordsCS{Nahraďte seznamem klíčových slov v češtině oddělených čárkou.}
\keywordsEN{Nahraďte seznamem klíčových slov v angličtině oddělených čárkou.}


\begin{document}

% TODO: Vyresit, jak budu formatovat odstavce.
% TODO: Zjistit, jak se vkladaji citace.

% \newacronym{CVUT}{{\v C}VUT}{{\v C}esk{\' e} vysok{\' e} u{\v c}en{\' i} technick{\' e} v Praze}
% \newacronym{FIT}{FIT}{Fakulta informa{\v c}n{\' i}ch technologi{\' i}}

\begin{introduction}
	%sem napište úvod Vaší práce
\end{introduction}

\chapter{Cíl práce}

Cílem mé bakalářské práce je vytvořit backend pro webovou aplikaci, která bude zajišťovat online
skórování zápasů a následné zobrazování statistik. Sou\-částí práce tak bude analýza dosavadních
i možných řešení. Výsledkem analýzy pak bude návrh a implementace webové aplikace. Frontend bude
vyvíjen v~rámci souběžné bakalářské práce Jaroslava Veselého. Mým úkolem je vytvořit backend,
jenž bude obsluhovatelný pomocí REST API. Vznikající systém má pracovní název Catcher.

%- Uvod do problematiky (Analyza problematiky)
%-- Co je to Ultimate Frisbee
%-- Uvod do aplikacnich rozhrani
%--- API
%
%- Pozadavky a zadani
%- Existujici reseni (Analyza dostupnych sluzeb)
%-- Catcher
%-- UltimateCentral
%
%- Požadavky a zadání
%- Analýza dostupných služeb
%  - Catcher
%  - UltimateCentral
%- Analýza dostupných technologií
%  - Výběr a detailnější popis vybraných technologií.
%- Návrh aplikace (db, model, atd.)
%  - Podivat se na to, co rikal Hunka.
%- Problemy pri implementaci
%- ...

\chapter{Úvod do problematiky}

\indent

V první kapitole bychom se měli seznámit s pojmem Ultimate Frisbee. Dále pak nastíníme
problematiku pořádání turnajů v tomto minoritním sportu.

\section{Co je to Ultimate Frisbee}

% TODO: Tady si mozna napsat neco uz hotovy na netu. Zaroven by bylo dobry zminit, jak se ultimate hraje.
% TODO: Obrazek???

\indent

Ultimate Frisbee je mladý a dynamicky se rozvíjející sport s létajícím talířem,
který se hraje od roku 1968 \cite{cald-o-ultimate}. Hraje jej přibližně sedm miliónů
hráčů ve více než 80 zemích světa a jeho popularita rok od roku stoupá \cite{usa-about-ultimate}.
Během posledních několika let je například běžné sledovat živé přenosy na sportovním kanálu ESPN
z Amerických soutěží, především profesionální ligy AUDL. Ultimate v roce 2015 dokonce získalo
uznání od Mezinárodního olympijského výboru \cite{cald-uznani}. Nejčastěji se hraje v kategoriích
open (muži), ženy, mix (smíšené týmy mužů a žen), junioři (do 19 let) a masters (nad 33 let).

\subsection{Pravidla}

Stručně popsaná pravidla podle Česká asociace lé\-ta\-jícícho disku:

\begin{quote}
``Ultimate je kolektivní bezkontaktní sport, v němž vítězí tým, který má na konci hrací doby
vyšší počet bodů. Hraje se na hřišti o rozměrech cca 100x37 metrů (délka fotbalového hřiště,
polovina jeho šířky). Na obou koncích hřiště jsou vyznačeny koncové zóny o hloubce cca 18 metrů.

V ultimate proti sobě hrají dva sedmičlenné týmy. Smyslem hry je pomocí přihrávek dopravit disk
do soupeřovy koncové zóny a jeho chycením v zóně získat bod. Po chycení disku se hráč musí
zastavit a do 10 vteřin disk přihrát spoluhráči. Povoleným pohybem hráče s diskem je pivotování,
tedy otáčení se kolem vlastní osy s jednou nohou pevně na zemi. V ultimate hráči často střídají
útok a obranu při ztrátě disku, ke které dochází záhozem disku do autu, na zem, jeho zachycením
soupeřem nebo při dlouhém držení disku. Není povolen fyzický kontakt mezi hráči ani přetahování
o disk.	''
\end{quote}

% TODO: Mam zminit, ze se hraje 5 na 5?

\subsection{Spirit of the Game}

\indent

Už od počátku je Ultimate Frisbee založeno na sportovním duchu, který klade odpovědnost
za fair play na samotné hráče. Předpokládá se vysoce soutěživá hra, ne však za cenu ztráty
vzájemné ohleduplnosti a vytracení radosti ze hry. Všechny přetupky na hřišti i mimo něj jsou
řešeny samotnými hráči. Jako jediná sportovní hra se tak obejde bez rozhodčích, a to i
na nejvyšších soutěžích, kterými jsou mistrovství Evropy a světa.

\medskip

Hraní fairplay je otázka cti. Na každém turnaji je vyhlašována cena Spirit of the Game,
která je cenou pro ty, kteří se chovali nejčestněji. Po každém zápase se týmy navzájem ohodnotí
v podobě číselného hodnocení a cenu pak získá tým s nejvyšším průměrem. Cena Spirit of the Game
je ceněna obdobně jako 1. místo.

\subsection{Ultimate v České republice}

\indent

V České republice zastřešuje sporty s létajícím talířem již od roku XXXX Česká asociace
lé\-ta\-jícícho disku (dále jen ``ČALD''). V celé republice eviduje XX zaregistrovaných klubů,
které se mezi sebou utkávají nejčastěji na víkendových turnajích. Těch bylo na našem
území za loňský rok více jak třicet, což je prakticky každý druhý týden v roce.

\medskip

Většina turnajů nebo mistrovství pak trvá zpravidla více dnů, během kterých se odehrají desítky
utkání. A s přibývajícím počtem hráčů a fanoušků vzniká čím dál větší poptávka po online
přenosech a statistikách z těchto akcí.

% TODO: Bude potreba doplnit, protoze se jedna o dulezity pojem, ktery bude pak zpracovan.

\subsection{Jak probíhá typický turnaj}

% TODO: dokoncit

\indent

Několik dní před turnajem se zveřejní rozpis, zpravidla v podobě dokumentu na Google Drive apod.
Ten je pak během turnajem editovaný a slouží jako jediný zdroj výsledků, který ... 

Nejdůležitejším údajem jsou výsledky z jednotlivých zápasů. Ty jsou nejčastěji zapisovány
na papír, který je vystaven na viditelném místě, aby si jej mohlo prohlédnout co nejvíce lidí.

\medskip

Částečně tyto procesy nahrazuje mobilní aplikace Catcher, ke které se ještě dostaneme.

% Vsechno se doposud pise na papir a je to proste cely na prd.
% TODO: Kdo vsechno, kolik lidi, na turnaji pobyva. Kolik probiha zapasu, treba i paralelne, kdo se o nej stara (navrh ma mobilni app). Co vsechno se da vycist z vysledku.
% TODO: Jeste doplnit dal.

\section{REST API}


\chapter{Specifikace požadavků}

Zadání vzniklo původně z požadavku na rozšíření mobilní aplikace Catcher, ke které se ještě
dostaneme v analýze již fungujících řešení. Se vznikajícími nároky na rozšiřitelnost přišel
také požadavek na RESTful rozhraní, které by backend obsluhovalo. V tu chvíli jsem tu byl
již já s nápadem na vytvoření nového backendu plně ovladatelným pomocí REST API.

\section{Požadavky na Catchera}

\indent

Na základě diskuse s ČALD, častými organizátory turnajů a některými hráči byly stanoveny
funkční a nefunkční požadavky na vznikající aplikaci.

\subsection{Funkční požadavky}

% TODO: Vice se rozepsat u jednotlivych bodu.


\begin{description}
  \item[Import a export dat] \hfill \\
  % TODO: Tohle neni jiste, zda uplne souhlasi.
  Systém umožňuje import oddílů a jeho hráčů z databáze ČALD.

  \item[Rozpis zápasů] \hfill 
  Oprávněná role může vytvořit svůj vlastní turnaj a vložit seznam všech zápasů,
  které se budou hrát. Vytvořený rozpis zápasů systém již doplňuje automaticky na základě
  odehraných výsledků (např. vítěze semifinále automaticky posune do finále). Ve chvíli,
  kdy to bude již možné, doplní celkové pořadí turnaje. Automatické doplnění se týká i 
  souhrné tabulky vzájemných hodnocení v kategorii SOTG.

  \item[Zadávání dat v rámci turnaje] \hfill \\
  Každý tým má možnost vytvořit vlastní soupisku na turnaj, kde bude hrát. Systém pak umožňuje
  v průbehu turnaje zadávat konkrétní údaje:
  \begin{itemize}
    \item průběžné skóre zápasů nebo jejich závěrečný výsledek
    \item skórující a asistující hráč
    \item hodnocení SOTG
  \end{itemize}
  
  \item[Tvorba statistik] \hfill \\
  Systém tvoří detailní statistiky hráčů, týmů a zápasů (obdržené a udělené body,
  počet asistencí, průměrná hodnota SOTG).
\end{description}

\subsection{Nefunkční požadavky}

% Zdroj: https://cs.wikipedia.org/wiki/Nefunk%C4%8Dn%C3%AD_po%C5%BEadavky_softwarov%C3%A9_architektury
% TODO: Muze se pridat Spravovatelnost (monitoring a dynamicka konfigurace aplikace)

\begin{description}
  \item[Snadná rozšířitelnost] \hfill \\
  Už nyní evidujeme změny, o které je zájem, ale nejsou předmětem této práce.
  I proto je nutné projekt dokončit tak, aby byl v budoucnu snadno rozšiřitelný nebo
  modifikovatelný.

  \item[Nízká cena] \hfill \\
  Cílem není vytvořit výdělečný projekt, ale fungující službu pro několik stovek hráčů
  a fanoušku v České republice. I proto je požadavkem použití volně dostupných
  knihoven a technologií.

  \item[Výkon] \hfill \\
  Podle celkem jednoduchých odhadů lze usoudit, že aplikaci budou čekat výkyvy v provozu.
  Většina zápisů a čtění dat probíhá během samotných akcí. Podle statistických údajů
  z již užívané aplikace Catcher víme, že během špičky není počet požadavků za sekundu
  větší než několik desítek. I proto není na celkový výkon kladen žádný zvláštní požadavek.
  Systém by každopádně měl více jak 95 \% žádostí zpracovat do 2 sekund (XXX milisekund) a na žádnou z nich
  nedpovědět za více jak 10 sekund (milisekund??).

  \item[Spolehlivost] \hfill \\
  Spolehlivost je základem pro funkční běh Catchera. Velká část operací, včetně chybných budou
  zapisovány do souborů pro snadné odhalení chyb. V případě výpadku by měl být informován
  administrátor pomocí SMS nebo emailu. Obnova musí být proveditelná ze zálohovacích souborů.
  
  % TODO: O vypadku nijak neinformuju, zalohu neprovadim
  
  \item[Bezpečnost] \hfill \\
  % autentizace a autorizace
  Systém musí jednoznačně určit a ověřit uživatele, který přistupuje k rozhraní Catchera. Zároveň
  musí existovat možnost za chodu přidávat, odebírat nebo měnit oprávnění. Dále pak systém musí
  být datově integritní, tzn. že obsah zpráv nebude při přenosu změněn.

  \item[Nároky na hardware] \hfill \\
  Systém musí být schopen běžet na běžných serverech s ne více jak 4GB RAM.

  \item[Formát importu] \hfill \\
  Pro import soupisek musí systém umět číst data ve formátu, který ČALD pro export používá.
  
  \item[Vytvoření dokumentace] \hfill \\
  Pro potřeby vývoje frontendu je potřeba vytvořit dostatečně detailní dokumentaci rozhraní API.
\end{description}

\section{Uživatelské role}

% TODO: Kdo vsechno bude do aplikace pristupovat
% TODO: Asi bych mel poznamenat do pozadavku, ze na kazdy turnaj je jine heslo.

\begin{description}
  \item[Nepřihlášený návštěvník] \hfill \\
  Jde o nejčastější přístup k aplikaci. Slouží k zobrazení všech statistik (aktuální skóre,
  statistika všech hráčů, hodnocení SOTG). Nemůže žádná data vytvářet nebo modifikovat
  a nevyžaduje přihlášení.
  
  \item[Organizátor] \hfill \\
  Účet pod touto rolí může vytvořit kdokoliv pouhou registrací. Po příhlášení lze přídávat
  nové turnaje a ty následně editovat. Tím se rozumí tvorba rozpisu zápasů a seznamu účastníků
  z již existujícího seznamu týmů. Dále může zadávat průběžné a výsledné skóre zápasů,
  skórující a asistující hráče a vidí na tabulku vzájemných hodnocení SOTG pro účely vyhlášení
  vítěze. Tato tabulka je pro ostatní role až do vyhlášení nezobrazitelná.
  
  \item[Klubový/oddílový účet] \hfill \\
  Pro účely odevzdávání hodnocení SOTG a úprav v týmové soupisce je vytvořen pro každý oddíl
  jeden společný účet. Tento účet má možnost vytvořit pouze admin. Přihlašovací údaje pak
  získá zástupce klubu a je pouze na něm, kdo z jeho oddílu bude spravovat vytvořený učet.
  
  \item[Admin] \hfill \\
  Má plnou kontrolu nad správou účtů a nad daty v databázi. Jeho možnosti zahrnují možnosti
  všech ostatních rolí.
\end{description}

\section{Případy užití}

Následující část popisuje nejběžnější případy užití. V návrhu RESTful rozhraní jde
o jednu z nejdůležitejších součastí, protože určuje směr návrhu. Výsledné API by tak mělo
pokrývat všechny níže uvedené případy.
% TODO: Mozna napsat neco o tom, ze kazde spravne api se dela tak, ze se zjisti vsechny usecasy, aby se mohli vytvorit dobre api.
% TODO: Co budou nejbeznejsi operace v aplikaci (zapisovani skore, zapisovani spiritu, tvorba turnaje a rozpisu). Tady to mozna napsat VSECHNO v bodech, protoze z toho budou vychazet API.

Následující seznam zachycuje jednotlivé požadavky a jejich popis.

% TODO: Napsat, kdo vsechno muzes tyto operace delat. Napr. organizator muze ukoncit zapas.

\begin{description}
 \item[F1: Import dat z databáze ČALD] \hfill \\
 Doplňuje seznam oddílů a jejich hráčů do databáze Catchera. V pravidelném intervalu stáhne
 aktuální data z ČALD databáze a doplní doposud aktuální data v databázi.

 % TODO: neexistuje zadny automaticky import
 
 \item[Získání libovolného oddílu z databáze] \hfill \\
 Získá množinu oddílů, jež jsou v databázi uloženy...
 % TODO: dokoncit
 
 \item[Získání libovolného hráče z databáze] \hfill \\
 ...
 % TODO: dokoncit
 
 \item[Vytvoření turnaje] \hfill \\
 Vytvoří konkrétní podobu turnaje. Do databáze uloží účastníky turnaje, základní skupiny,
 zápasy ve skupinách a následnující zápasy v play-off.
 
 \item[Získání libovolného turnaje z databáze] \hfill \\
 Získá množinu turnajů, jež jsou v databázi uloženy. S pomocí několika parametrů lze tuto
 množinu filtrovat:
 \begin{itemize}
    \item turnaj s konkrétním identifikátorem
    \item pouze ne/ukončené turnaje
    \item pouze ne/aktivní turnaje
    \item turnaje odehrávající se v čase od do
    \item turnaje v kategoriích open, women, mix apod.
    \item turnaje vytvořené konkrétním uživatelem
  \end{itemize}
  Prvek množiny obsahuje informace o turnaji, jako například název, popis, termín a místo turnaje.
 
 \item[Získá účastníky turnaje] \hfill \\
 Ke konkrétnímu turnaji získá kolekci všech týmů, jež se účastní. 
 
 \item[Získá týmovou soupisku] \hfill \\
 Ke konkrétnímu týmu získá kolekci všech hráčů, jenž se účastní.
 
 \item[Zadání skórujícího nebo asistujícího hráče] \hfill \\
 V průběhu zápasu umožňuje zadat skórujícího nebo asistujícího hráče.
 
 \item[Ukončit zápas s výsledným skóre] \hfill \\
 Ukončí zápas a uloží jeho výsledek do databáze.
 
 \item[Zadat hodnocení SOTG] \hfill \\
 Po ukončení zápasu lze z týmového účtu udělit soupeři hodnocení SOTG. Tento údaj se uloží do tabulky, kde jsou všechna vzájemná hodnocení. 
 
 \item[Zobrazit tabulku vzájemných hodnocení SOTG] \hfill \\
 Odešle uživateli vhodnou strukturu, kde budou jasně uvedeny všechna doposud vyplněná hodnocení kategorie SOTG.
 
 \item[Zobrazit pořadí týmů v kategorii SOTG] \hfill \\
 Vrátí pořadí týmů v kategorii SOTG, včetně průměru všech obdržených hodnocení, a informaci o tom, zda je toto pořadí již finální.
 
 \item[Ukončit turnaj] \hfill \\
 Změní stav turnaje na ukončený. Tabulka vzájemných hodnocení SOTG a výsledné pořadí se tím stanou veřejnými. 
 
 \item[Zobrazit výsledky všech zápasů] \hfill \\
 
 \item[Zobrazit detail konkrétního zápasu] \hfill \\ hráči, body, čas
 
 \item[Získá seřazené hráče podle bodování] \hfill \\ Seřadí hráče na základě bodovaní.
\end{description}

\chapter{Analýza}

\section{Existující řešení}

% TODO: Jake reseni uz existuji. Catcher, UltimateCentral atd., co je na nich dobryho a co ne

\section{Možnosti řešení}

% TODO: Co vsechno si z techto aplikaci muzu vzit. Pouzit API nebo jine technologie.

\section{Něco o API}

% TODO: Napsat neco o teorii REST API. Mozna lze sjednotit s predchozi kapitolou.

\chapter{Návrh}

\section{Návrh tříd a databáze}

% TODO: Takovy ty kecy teoreticky. Jak ma vypadat diagram trid, databazovy model apod.

\section{Architektura a návrh nasazení}

% TODO: Nevim, ale asi jak se to cely nasadi.

\chapter{Implementace}

\section{Výběr technologií}

\indent

V době, kdy existuje obrovské množství programovacích jazyků, jejich ekosystémů a dalších frameworků je důležité se umět nespálit. 
Vybrat špatné technologie totiž může znamenat prodloužení vývoje, zvýšení celkových nákladů nebo dokonce ukončení projektu.
Proto je nutnost tuto část dostatečně analyzovat a svědomitě zvážit, aby nedošlo k omylu, který by zabraňoval úspěšné dokončení.
Pro adekvátní výběr tak bylo potřeba specifikovat příslušná kritéria, které by jasně porovnala všechny zvažované technologie:

\subsubsection*{Podpora webových služeb}
Při tvorbě webové aplikace nemůže být ani řeč o technologii, která by pro použití na webu nebyla vhodná. Technologie musí
umět pracovat s mapováním URL a HTTP metod na zdroje a pracovat s běžnými formáty dat v internetové komunikaci (např. JSON).

\subsubsection*{Dokumentace a uživatelská podpora}
Pro pochopení technologie a její správné používání je často potřeba znát detaily, které se uživatel dozví z dokumentace.
Ta dokáže často výrazně zkrátit učící křivku a programátorovi tak šetří čas. Pokud dokumentace nestačí, může často její
nedostatky zachránit dostatečně široká uživatelská základna. Velká uživatelská podpora navíc snižuje riziko zestárnutí zvolené technologie.

\subsubsection*{Znalost technologie}
Pokud již programátor některou z technologií zná, může mu tato znalost ušetřit mnoho času při vývoji. Nemusí totiž novou technologii
nijak složitě zavádět nebo se učit její použití. Vhodným výběrem se také může vyvarovat situacím, kdy až v průběhu implementace narazí
na neřešitelné problémy. Zavrhnout neznámé technologie ale může znamenat přijít o možnost používat nástroje, které jsou problémům šité na míru. 

\subsubsection*{Výkon}
I když aplikace nepočítá s vetším provozem čítající stovky požadavků za sekundu, není důvod používat technologie, které jsou pomalé
nebo spotřebovávají příliš moc prostředků.

\section{Programovací jazyk Python}
% TODO: zdroj k letopoctu: https://cs.wikipedia.org/wiki/Python#cite_note-1
Python je víceúčelový skriptovací jazyk navrhnut v roce 1991 Guido van Rossumem. 

% TODO: napsat si vsechny zajimave informace do bodu a pak to nejak poskladat

Python je vyvíjen jako open source, nabizi baliky pro unix, windows a mac os, ve vetsine Linuxovych distribuci je soucasti zakladni instalace
Hybridni jazyk, ...
% TODO: napsat historii a vsechno o pythonu

\section{Webový framework Falcon}

\section{Databáze MySQL}

\section{Peewee ORM}

\section{Webový server Nginx}

\section{Uwsgi}

% TODO: Jake technologie si vyberu. Ze kterych mam na vyber.

% TODO: Napsat neco o Percone (MySQL), Falconu, Pythonu a mozna necem dalsim.

\chapter{Testování}

\indent

Jednou z nejdůležitejších součástí vývoje je testování. Pomáhá zkoumat funkčnost softwaru a v případě bezchybného
a úplného testování zajišťuje, že výsledná aplikace neobsahuje chyby a má požadované vlastnosti.
Testování se často provádí ve více iteracích, například pravidelně před vydáním nové verze softwaru.
Součástí testování je pak především reportování nalezených chyb.

\medskip

Testy se dělí na tzv. black-box a white-box testy. Při testování černé skřínky nemáme informace o vnitřní podobě a porovnáváme
pouze správnost výstupních dat po zadání vstupních. Do testovaného subjektu se nelze podívat, vídíme jen to, jak se chová navenek.
Důvodem tohoto typu testování je analyzovat software z pohledu uživatele.
Na opačné straně se při znalosti implementace používá testování bílé skřínky. Sice tato metoda zastiňuje pohled uživatele,
ale tester je schopen lépe určit, kde hledat případné chyby. Při~částečné znalosti softwaru se ještě používá pojem testování šedé skřínky (gray box).
Neznáme například přesnou podobu zdrojového kódu, ale jen použitý algoritmus použitý v aplikaci. 

% TODO: moznost se zde rozepsat s rozdelenim dalsich testu
% Další možností, jak rozdělovat testy je rozdělení dle...

\medskip

% TODO: napsat, co je to manualni testovani
Testy se také dělí na automatizované a manuální. Méně nákladnou a rychlejší cestou je zpravidla automatizované testování,
kdy se automaticky spouští velké množství testů, často i s velkým množstvím vstupních dat. Tyto testy se používají v místech,
kde dochází k opakování stejného nebo velmi podobného scénáře. Ne vždy je ale jejich použití možné, test vyžaduje lidský úsudek
nebo rozdílné přístupy a testy není nutné pravidelně opakovat, a proto se používá kombinace obojího. 

\section{Testování Catchera}

\indent

Testování softwaru je stále velmi podceňovaný obor a často je tato fáze vývoje odsunuta na druhou kolej nebo zcela vynechána.
Kromě toho, že psaní testů se ukázalo jako nejrychlejší způsob odhalování chyb, bylo pak ještě více překvapující, že při~použití
techniky TDD\footnote{Test-driven development - technika, kdy se testy píší ještě před samotným vývojem.} se znatelně zefektivnil
vývoj. Už při psaní zdrojového kódu jsem tak měl zpětnou vazbu a daleko dříve jsem mohl ukončit tvorbu některých komponent.
Díky specifickému popisu získání a manipulace se zdroji v architektuře REST jsem většinu testů podřídil tomuto rozdělení a prováděl
jsem tzv. testování funkcionalit - z pohledu uživatele, ověřování úzce zaměřených scénářů.

\medskip

V Pythonu se nejrozšířenější framwork pro automatizované testování jmenuje Unittest. Já jsem však využil testovací nástroj, jenž přímo
poskytuje webový framework Falcon. Je potřeba zmínit, že tento nástroj většinu metod dědí od základního Unittestu a jeho možnosti pouze
rozšiřuje pro lepší a rychlejší testování REST API.

\medskip

Automatizovanými testy jsou v Catcheru pokryty všechny zdroje s jejich metodami. Typicky jsou testy po několika sdružovány do jedné třídy,
ve které je možnost vytvořit specifické testovací prostředí. V nástroji Unittest jsou pro tyto účely často používány metody \pythoninline{setUp()} a \pythoninline{tearDown()},
které jsou volány před, respektive po každém testu. Důsledek tohoto chování je tak zajištění nezávislosti na dalších testech.
Kdyby se tak nestalo, tak například v testech, které modifikují testovací data, by mohlo dojít k tomu, že při pouhém prohození ve frontě testů
bude test pracovat s odlišnou množinou dat a vracet jiné výsledky, než byly očekávány v původním pořadí fronty. Nejčastější podoba testů pro Catchera
je vidět v následující velmi zjednodušené ukázce kódu: 

\begin{python}
class TestClubs(falcon.testing.TestBase):

    def setUp(self):
      '''
      Spustí se před každým testem,
      zpravidla naplní testovací databázi daty.
      '''

    def tearDown(self):
      '''
      Spustí se po každém testu,
      zpravidla vyčistí testovací databázi.
      '''

    def testGet(self):
      '''Testuje požadavek GET - získá kolekci všech oddílů.'''     
      response = self.simulate_request('GET', '/api/clubs')
      self.assertEqual(self.srmock.status, HTTP_200)

    def testPost(self):
      '''Testuje požadavek POST - vytvoří nový oddíl.'''
      response = self.simulate_request('POST', '/api/clubs', body)
      self.assertEqual(self.srmock.status, HTTP_201)
\end{python}

Kromě porovnávání návratových HTTP kódů se v mým testech zkoumá podoba a obsah návratových dat, správné vyhazování vyjímek
nebo stav testovací databáze před a po volání metod.

% TODO: IDEA: Lze se zminit o testovani a prubezne integraci. Napriklad napsat, ze mame zajem v budoucnu pouzivat Codevoc nebo Travis CI.

\chapter{Nasazení}

% TODO: Popis jake technologie jsem si vybral na provoz aplikace na serveru: web server a uwsgi + uml diagram nasazeni

\section{Softwarové požadavky}

\section{Konfigurace}

\section{Údržba}

XXXXXXXXXXx \cite{JJ92}

% \chapter{Analýza a návrh}
% \chapter{Realizace}

\begin{conclusion}
	%sem napište závěr Vaší práce
	% TODO: Budoucnost projektu.
\end{conclusion}

% TODO: Promyslet, jak presne odevzdavat citace.
%\bibliographystyle{csn690}
%\bibliographystyle{iso690}
%\bibliography{ref}

\appendix

\chapter{Seznam použitých zkratek}
% \printglossaries
\begin{description}
	\item[GUI] Graphical user interface
	\item[XML] Extensible markup language
	\item[ČALD] Česká asociace létajícího disku
	\item[SOTG] Spirit of the Game
	\item[REST] Representational State Transfer
\end{description}

\chapter{Obsah přiloženého CD}

%upravte podle skutecnosti

\begin{figure}
	\dirtree{%
		.1 readme.txt\DTcomment{stručný popis obsahu CD}.
		.1 exe\DTcomment{adresář se spustitelnou formou implementace}.
		.1 src.
		.2 impl\DTcomment{zdrojové kódy implementace}.
		.2 thesis\DTcomment{zdrojová forma práce ve formátu \LaTeX{}}.
		.1 text\DTcomment{text práce}.
		.2 thesis.pdf\DTcomment{text práce ve formátu PDF}.
		.2 thesis.ps\DTcomment{text práce ve formátu PS}.
	}
\end{figure}

\end{document}

\iffalse
\fi