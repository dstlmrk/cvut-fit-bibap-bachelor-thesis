% options:
% thesis=B bachelor's thesis
% thesis=M master's thesis
% czech thesis in Czech language
% slovak thesis in Slovak language
% english thesis in English language
% hidelinks remove colour boxes around hyperlinks


\documentclass[thesis=B,czech]{FITthesis}[2012/06/26]

\usepackage[utf8]{inputenc} % LaTeX source encoded as UTF-8

\usepackage{graphicx} %graphics files inclusion
% \usepackage{amsmath} %advanced maths
% \usepackage{amssymb} %additional math symbols

\usepackage{dirtree} %directory tree visualisation



% % list of acronyms
% \usepackage[acronym,nonumberlist,toc,numberedsection=autolabel]{glossaries}
% \iflanguage{czech}{\renewcommand*{\acronymname}{Seznam pou{\v z}it{\' y}ch zkratek}}{}
% \makeglossaries

\newcommand{\tg}{\mathop{\mathrm{tg}}} %cesky tangens
\newcommand{\cotg}{\mathop{\mathrm{cotg}}} %cesky cotangens

% % % % % % % % % % % % % % % % % % % % % % % % % % % % % % 
% ODTUD DAL VSE ZMENTE
% % % % % % % % % % % % % % % % % % % % % % % % % % % % % % 

\department{Katedra softwarového inženýrství}
\title{Systém pro skórování Ultimate~Frisbee~zápasů~-~backend}
\authorGN{Marek} %(křestní) jméno (jména) autora
\authorFN{Dostál} %příjmení autora
\authorWithDegrees{Marek Dostál} %jméno autora včetně současných akademických titulů
\supervisor{Ing. Jiří Hunka}
\acknowledgements{Doplňte, máte-li komu a za co děkovat. V~opačném případě úplně odstraňte tento příkaz.}
\abstractCS{V~několika větách shrňte obsah a přínos této práce v~češtině. Po přečtení abstraktu by se čtenář měl mít čtenář dost informací pro rozhodnutí, zda chce Vaši práci číst.}
\abstractEN{Sem doplňte ekvivalent abstraktu Vaší práce v~angličtině.}
\placeForDeclarationOfAuthenticity{V~Praze}
\declarationOfAuthenticityOption{4} %volba Prohlášení (číslo 1-6)
\keywordsCS{Nahraďte seznamem klíčových slov v češtině oddělených čárkou.}
\keywordsEN{Nahraďte seznamem klíčových slov v angličtině oddělených čárkou.}


\begin{document}

% TODO: Vyresit, jak budu formatovat odstavce.
% TODO: Zjistit, jak se vkladaji citace.

% \newacronym{CVUT}{{\v C}VUT}{{\v C}esk{\' e} vysok{\' e} u{\v c}en{\' i} technick{\' e} v Praze}
% \newacronym{FIT}{FIT}{Fakulta informa{\v c}n{\' i}ch technologi{\' i}}

\begin{introduction}
	%sem napište úvod Vaší práce
\end{introduction}

\chapter{Cíl práce}

Cílem mé bakalářské práce je vytvořit backend pro webovou aplikaci, která bude zajišťovat online
skórování zápasů a následné zobrazování statistik. Sou\-částí práce tak bude analýza dosavadních
i možných řešení. Výsledkem analýzy pak bude návrh a implementace webové aplikace. Frontend bude
vyvíjen v~rámci souběžné bakalářské práce Jaroslava Veselého. Jeho úkolem je vytvořit především
uživatelsky přívětivé prostředí, které bude pohodlné pro správu na mobilních zařízeních. Jeho
výstupem bude tedy responzivní web. Vznikající systém má pracovní název UFSS.

%- Uvod do problematiky (Analyza problematiky)
%-- Co je to Ultimate Frisbee
%-- Uvod do aplikacnich rozhrani
%--- API
%
%- Pozadavky a zadani
%- Existujici reseni (Analyza dostupnych sluzeb)
%-- Catcher
%-- UltimateCentral
%
%- Požadavky a zadání
%- Analýza dostupných služeb
%  - Catcher
%  - UltimateCentral
%- Analýza dostupných technologií
%  - Výběr a detailnější popis vybraných technologií.
%- Návrh aplikace (db, model, atd.)
%  - Podivat se na to, co rikal Hunka.
%- Problemy pri implementaci
%- ...

\chapter{Úvod do problematiky}

\indent

V první kapitole bychom se měli seznámit s pojmem Ultimate Frisbee. Dále pak nastíníme
problematiku pořádání turnajů v tomto minoritním sportu.

\section{Co je to Ultimate Frisbee}

% TODO: Tady si mozna napsat neco uz hotovy na netu. Zaroven by bylo dobry zminit, jak se ultimate hraje.
% TODO: Obrazek???

\indent

Ultimate Frisbee je mladý a dynamicky se rozvíjející sport s létajícím talířem,
který se hraje od roku 1968 \cite{cald-o-ultimate}. Hraje jej přibližně sedm miliónů
hráčů ve více než 80 zemích světa a jeho popularita rok od roku stoupá \cite{usa-about-ultimate}.
Během posledních několika let je například běžné sledovat živé přenosy na sportovním kanálu ESPN
z Amerických soutěží, především profesionální ligy AUDL. Ultimate v roce 2015 dokonce získalo
uznání od Mezinárodního olympijského výboru \cite{cald-uznani}. Nejčastěji se hraje v kategoriích
open (muži), ženy, mix (smíšené týmy mužů a žen), junioři (do 19 let) a masters (nad 33 let).

\subsection{Pravidla}

Stručně popsaná pravidla podle Česká asociace lé\-ta\-jícícho disku:

\begin{quote}
``Ultimate je kolektivní bezkontaktní sport, v němž vítězí tým, který má na konci hrací doby
vyšší počet bodů. Hraje se na hřišti o rozměrech cca 100x37 metrů (délka fotbalového hřiště,
polovina jeho šířky). Na obou koncích hřiště jsou vyznačeny koncové zóny o hloubce cca 18 metrů.

V ultimate proti sobě hrají dva sedmičlenné týmy. Smyslem hry je pomocí přihrávek dopravit disk
do soupeřovy koncové zóny a jeho chycením v zóně získat bod. Po chycení disku se hráč musí
zastavit a do 10 vteřin disk přihrát spoluhráči. Povoleným pohybem hráče s diskem je pivotování,
tedy otáčení se kolem vlastní osy s jednou nohou pevně na zemi. V ultimate hráči často střídají
útok a obranu při ztrátě disku, ke které dochází záhozem disku do autu, na zem, jeho zachycením
soupeřem nebo při dlouhém držení disku. Není povolen fyzický kontakt mezi hráči ani přetahování
o disk.	''
\end{quote}

% TODO: Mam zminit, ze se hraje 5 na 5?

\subsection{Spirit of the Game}

\indent

Už od počátku je Ultimate Frisbee založeno na sportovním duchu, který klade odpovědnost
za fair play na samotné hráče. Předpokládá se vysoce soutěživá hra, ne však za cenu ztráty
vzájemné ohleduplnosti a vytracení radosti ze hry. Všechny přetupky na hřišti i mimo něj jsou
řešeny samotnými hráči. Jako jediná sportovní hra se tak obejde bez rozhodčích, a to i
na nejvyšších soutěžích, kterými jsou mistrovství Evropy a světa.

\medskip

Hraní fairplay je otázka cti. Na každém turnaji je vyhlašována cena Spirit of the Game,
která je cenou pro ty, kteří se chovali nejčestněji. Po každém zápase se týmy navzájem ohodnotí
v podobě číselného hodnocení a cenu pak získá tým s nejvyšším průměrem. Cena Spirit of the Game
je ceněna obdobně jako 1. místo.

\subsection{Ultimate v České republice}

\indent

V České republice zastřešuje sporty s létajícím talířem již od roku XXXX Česká asociace
lé\-ta\-jícícho disku (dále jen ``ČALD''). V celé republice eviduje XX zaregistrovaných klubů,
které se mezi sebou utkávají nejčastěji na víkendových turnajích. Těch bylo na našem
území za loňský rok více jak třicet, což je každý druhý týden v roce.

\medskip

Většina turnajů nebo mistrovství pak trvá zpravidla více dnů, během kterých se odehrají desítky
utkání. A s přibývajícím počtem hráčů a fanoušků vzniká čím dál větší poptávka po online
přenosech a statistikách z těchto akcí.

% TODO: Bude potreba doplnit, protoze se jedna o dulezity pojem, ktery bude pak zpracovan.

\section{Jak probíhá typický turnaj}

\indent

Nejdůležitejším údajem jsou výsledky z jednotlivých zápasů. Ty jsou nejčastěji zapisovány
na papír, který je vystaven na viditelném místě, aby si jej mohlo prohlédnout co nejvíce lidí.

\medskip

Částečně tyto procesy nahrazuje mobilní aplikace Catcher, ke které se ještě dostaneme.

% Vsechno se doposud pise na papir a je to proste cely na prd.
% TODO: Kdo vsechno, kolik lidi, na turnaji pobyva. Kolik probiha zapasu, treba i paralelne, kdo se o nej stara (navrh ma mobilni app). Co vsechno se da vycist z vysledku.
% TODO: Jeste doplnit dal.

\chapter{Specifikace požadavků}

Zadání vzniklo původně z požadavku na rozšíření mobilní aplikace Catcher, ke které se ještě
dostaneme v analýze již fungujících řešení. Se vznikajícími nároky na rozšiřitelnost přišel
také požadavek na RESTful rozhraní, které by backend obsluhovalo. V tu chvíli jsem tu byl
již já s nápadem na vytvoření nového backendu plně ovladatelným pomocí REST API.

\section{Požadavky na UFSS}

\indent

Na základě debaty s ČALD, častými organizátory turnajů a některými hráči byly stanoveny
funkční a nefunkční požadavky na vznikající aplikaci.

\subsection{Funkční požadavky}

% TODO: Vice se rozepsat u jednotlivych bodu.


\begin{description}
  \item[Import a export dat] \hfill \\
  % TODO: Tohle neni jiste, zda uplne souhlasi.
  Systém umožňuje import a export dat (týmy, hráči, soupisky). Navíc bude automaticky
  importovat data z databáze ČALD. Půjde o soupisky odevzdané jednotlivými týmy
  před turnaji, které to vyžadují.

  \item[Rozpis zápasů] \hfill \\
  % TODO: Nemam vymyslene, jak budu turnaje vytvaret.
  Oprávněná role může vytvořit svůj vlastní turnaj a vložit seznam všech zápasů,
  které se budou hrát. Vytvořený rozpis zápasů systém již doplňuje automaticky na základě
  odehraných výsledků (např. vítěze semifinále automaticky posune do finále). Ve chvíli,
  kdy to bude již možné, doplní celkové pořadí. To se týká i tabulky vzájemných hodnocení
  v kategorii SOTG.

  \item[Zadávání dat v rámci turnaje] \hfill \\
  Každý tým má možnost vytvořit vlastní soupisku na turnaj, kde bude hrát. Systém pak umožňuje
  v průbehu turnaje zadávat konkrétní údaje:
  \begin{itemize}
    \item průběžné skóre zápasů a jejich závěrečný výsledek
    \item skórující a asistující hráč
    \item hodnocení SOTG
  \end{itemize}
  
  \item[Tvorba statistik] \hfill \\
  Systém tvoří detailní statistiky hráčů, týmů a zápasů (obdržené a udělené body,
  počet asistencí, průměrná hodnota SOTG).
  
  \item[Notifikace] \hfill \\
  Systém rozesílá notifikace (pomocí SMS nebo emailu) všem nebo vybraným účastníkům
  turnaje z důvodu závažných událostí (odložení zápasu apod.).
\end{description}

% Specifikum této aplikace bude to, že výsledky v průběhu zápasů budou moci zapisovat i hráči,
% kteří jsou se svým mobilním zařízením na místě. Tato vlastnost má přinést rychlejší přístup
% k výsledkům na turnajích, na kterých není dostatečný počet organizátorů. Některé turnaje si
% například organizují samotní hráči.

\subsection{Nefunkční požadavky}

% Zdroj: https://cs.wikipedia.org/wiki/Nefunk%C4%8Dn%C3%AD_po%C5%BEadavky_softwarov%C3%A9_architektury
% TODO: Muze se pridat Spravovatelnost (monitoring a dynamicka konfigurace aplikace)

\begin{description}
  \item[Snadná rozšířitelnost] \hfill \\
  Už nyní evidujeme změny, o které je zájem, ale nejsou předmětem této práce.
  I proto je nutné projekt dokončit tak, aby byl v budoucnu snadno rozšiřitelný nebo
  modifikovatelný.

  \item[Nízká cena] \hfill \\
  Cílem není vytvořit výdělečný projekt, ale fungující službu pro několik stovek hráčů
  a fanoušku v České republice. I proto je požadavkem použití volně dostupných
  knihoven a technologií.

  \item[Výkon] \hfill \\
  Podle celkem jednoduchých odhadů lze usoudit, že aplikaci budou čekat výkyvy v provozu.
  Většina zápisů a čtění dat probíhá během samotných akcí. Podle statistických údajů
  z již užívané aplikace Catcher víme, že během špičky není počet požadavků za sekundu
  větší než několik desítek. I proto není na celkový výkon kladen žádný zvláštní požadavek.
  Systém by každopádně měl více jak 95 \% žádostí zpracovat do 2 sekund a na žádnou z nich
  nedpovědět za více jak 10 sekund.

  \item[Spolehlivost] \hfill \\
  Spolehlivost je základem pro funkční běh UFSS. Velká část operací, včetně chybných budou
  zapisovány do souborů pro snadné odhalení chyb. V případě výpadku by měl být informován
  administrátor pomocí SMS nebo emailu. Obnova musí být proveditelná ze zálohovacích souborů.

  \item[Bezpečnost] \hfill \\
  % autentizace a autorizace
  Systém musí jednoznačně určit a ověřit uživatele, který přistupuje k UFSS. Zároveň musí
  existovat možnost za chodu přidávat, odebírat nebo měnit oprávnění. Dále pak systém musí
  být datově integritní, tzn. že obsah zpráv nebude při přenosu změněn.

  \item[Nároky na hardware] \hfill \\
  Systém musí být schopen běžet na běžných serverech s ne více jak 4GB RAM.

  \item[Formát importu] \hfill \\
  Pro import soupisek musí systém umět číst data ve formátu, který ČALD pro export používá.
\end{description}

\section{Uživatelské role}

% TODO: Kdo vsechno bude do aplikace pristupovat
% TODO: Asi bych mel poznamenat do pozadavku, ze na kazdy turnaj je jine heslo.

\begin{description}
  \item[Nepřihlášený návštěvník] \hfill \\
  Jde o nejčastější přístup k aplikaci. Slouží k zobrazení všech statistik (aktuální skóre,
  statistika všech hráčů, hodnocení SOTG). Nemůže žádná data vytvářet nebo modifikovat
  a nevyžaduje přihlášení.
  
  \item[Organizátor] \hfill \\
  Účet pod touto rolí může vytvořit kdokoliv pouhou registrací. Po příhlášení lze přídávat
  nové turnaje a ty následně editovat. Tím se rozumí tvorba rozpisu zápasů a seznamu účastníků
  z již existujicího seznamu týmů (v případě potřeby nový tým vytvoří). Dále může zadávat
  průběžné a výsledné skóre zápasů, skórující a asistující hráče a vidí na vzájemnou tabulku
  hodnocení SOTG pro účely vyhlášení vítěze. Tato tabulka je pro ostatní role až do vyhlášení
  nezobrazitelná.
  
  \item[Účastník turnaje] \hfill \\
  % TODO: Kazdy, kdo ma heslo, se muze prihlasit a vysledky turnaje editovat. Je na organizatorovi, zda da tuto volbu i hracum (treba jenom nekterym).
  Tato role je doplňkem role Organizátor. Nemá všechna jeho práva, ale podstatně pomáhá
  tvorbě statistik na každém turnaji. Může totiž zapisovat průběžné a výsledné skóre u všech
  zápasů i skórující a asistující hráče. Účet je vytvořen vždy společně s turnajem.
  Přihlašovací údaje získá organizátor po vytvoření turnaje a je pouze na něm, komu tyto
  údaje předá. Sám tak volí, zda do tvorby statistik zasáhne někdo další nebo přenechá
  tuto možnost pouze sám sobě. Častým případem je rozeslání přihlašovacích údajů emailem všem
  účastníkům turnaje chvíli před turnajem.
  
  \item[Týmový účet] \hfill \\
  Pro účely odevzdávání hodnocení SOTG a úprav v týmové soupisce je vytvořen pro každý tým
  jeden společný účet. Tento účet má možnost vytvořit pouze admin nebo organizátor.
  Přihlašovací údaje pak získá zástupce týmu a je pouze na něm, kdo z jeho týmu bude
  spravovat týmový učet.
  
  \item[Admin] \hfill \\
  Má plnou kontrolu nad správou účtů a nad daty v databázi.  
\end{description}

\section{Případy užití}

Následující část popisuje nejběžnější (většinu?) případy užití. V návrhu RESTful rozhraní jde
o jednu z nejdůležitejších součastí, protože určuje směr návrhu. Výsledné API by tak mělo
pokrývat všechny níže uvedené případy.
% TODO: Mozna napsat neco o tom, ze kazde spravne api se dela tak, ze se zjisti vsechny usecasy, aby se mohli vytvorit dobre api.
% TODO: Co budou nejbeznejsi operace v aplikaci (zapisovani skore, zapisovani spiritu, tvorba turnaje a rozpisu). Tady to mozna napsat VSECHNO v bodech, protoze z toho budou vychazet API.

Přehled jednotlivých případů užití, které API umožňuje.

\begin{enumerate}
\item bla, bla
\end{enumerate}

\chapter{Analýza}

\section{Existující řešení}

% TODO: Jake reseni uz existuji. Catcher, UltimateCentral atd., co je na nich dobryho a co ne

\section{Možnosti řešení}

% TODO: Co vsechno si z techto aplikaci muzu vzit. Pouzit API nebo jine technologie.

\section{Něco o API}

% TODO: Napsat neco o teorii REST API. Mozna lze sjednotit s predchozi kapitolou.

\chapter{Návrh}

\section{Návrh tříd a databáze}

% TODO: Takovy ty kecy teoreticky. Jak ma vypadat diagram trid, databazovy model apod.

\section{Architektura a návrh nasazení}

% TODO: Nevim, ale asi jak se to cely nasadi.

\chapter{Implementace}

\section{Výběr technologií}

% TODO: Jake technologie si vyberu. Ze kterych mam na vyber.

% TODO: Napsat neco o Percone (MySQL), Falconu, Pythonu a mozna necem dalsim.

\chapter{Testování}

% TODO: Co budu pouzivat na testovani a prubeznou integraci. Codevoc nebo Travis CI.
% TODO: Jake testy budu delat (unit, akceptaci)

\chapter{Nasazení}

% TODO: Tady jeste uvidime, co napiseme.

\section{Softwarové požadavky}

\section{Konfigurace}

\section{Údržba}

XXXXXXXXXXx \cite{JJ92}

% \chapter{Analýza a návrh}
% \chapter{Realizace}

\begin{conclusion}
	%sem napište závěr Vaší práce
	% TODO: Budoucnost projektu.
\end{conclusion}

% TODO: Promyslet, jak presne odevzdavat citace.
%\bibliographystyle{csn690}
%\bibliographystyle{iso690}
%\bibliography{ref}

\appendix

\chapter{Seznam použitých zkratek}
% \printglossaries
\begin{description}
	\item[GUI] Graphical user interface
	\item[XML] Extensible markup language
	\item[ČALD] Česká asociace létajícího disku
	\item[SOTG] Spirit of the Game
\end{description}

\chapter{Obsah přiloženého CD}

%upravte podle skutecnosti

\begin{figure}
	\dirtree{%
		.1 readme.txt\DTcomment{stručný popis obsahu CD}.
		.1 exe\DTcomment{adresář se spustitelnou formou implementace}.
		.1 src.
		.2 impl\DTcomment{zdrojové kódy implementace}.
		.2 thesis\DTcomment{zdrojová forma práce ve formátu \LaTeX{}}.
		.1 text\DTcomment{text práce}.
		.2 thesis.pdf\DTcomment{text práce ve formátu PDF}.
		.2 thesis.ps\DTcomment{text práce ve formátu PS}.
	}
\end{figure}

\end{document}

\iffalse
\fi