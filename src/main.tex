% options:
% thesis=B bachelor's thesis
% thesis=M master's thesis
% czech thesis in Czech language
% slovak thesis in Slovak language
% english thesis in English language
% hidelinks remove colour boxes around hyperlinks


\documentclass[thesis=B,czech]{FITthesis}[2012/06/26]

\usepackage[utf8]{inputenc} % LaTeX source encoded as UTF-8

\usepackage{graphicx} %graphics files inclusion
% \usepackage{amsmath} %advanced maths
% \usepackage{amssymb} %additional math symbols

\usepackage{dirtree} %directory tree visualisation

% --------------------------------------------------------------------
% Default fixed font does not support bold face
\DeclareFixedFont{\ttb}{T1}{txtt}{bx}{n}{9} % for bold
\DeclareFixedFont{\ttm}{T1}{txtt}{m}{n}{9}  % for normal
% Custom colors
\usepackage{color}
\definecolor{deepblue}{rgb}{0,0,0.5}
\definecolor{deepred}{rgb}{0.6,0,0}
\definecolor{deepgreen}{rgb}{0,0.5,0}
\usepackage{listings}
\lstset{
     literate=%
         {á}{{\'a}}1
         {í}{{\'i}}1
         {é}{{\'e}}1
         {ý}{{\'y}}1
         {ú}{{\'u}}1
         {ó}{{\'o}}1
         {ě}{{\v{e}}}1
         {š}{{\v{s}}}1
         {č}{{\v{c}}}1
         {ř}{{\v{r}}}1
         {ž}{{\v{z}}}1
         {ď}{{\v{d}}}1
         {ť}{{\v{t}}}1
         {ň}{{\v{n}}}1                
         {ů}{{\r{u}}}1
         {Á}{{\'A}}1
         {Í}{{\'I}}1
         {É}{{\'E}}1
         {Ý}{{\'Y}}1
         {Ú}{{\'U}}1
         {Ó}{{\'O}}1
         {Ě}{{\v{E}}}1
         {Š}{{\v{S}}}1
         {Č}{{\v{C}}}1
         {Ř}{{\v{R}}}1
         {Ž}{{\v{Z}}}1
         {Ď}{{\v{D}}}1
         {Ť}{{\v{T}}}1
         {Ň}{{\v{N}}}1                
         {Ů}{{\r{U}}}1    
}
% Python style for highlighting
\newcommand\pythonstyle{\lstset{
language=Python,
basicstyle=\ttm,
otherkeywords={self},             % Add keywords here
keywordstyle=\ttb\color{deepblue},
emph={TestClubs,__init__,setUp,tearDown,testGet,testPost},          % Custom highlighting
emphstyle=\ttb\color{deepred},    % Custom highlighting style
stringstyle=\color{deepgreen},
% frame=tb,                         % Any extra options here
showstringspaces=false            % 
}}
% Python environment
\lstnewenvironment{python}[1][]
{
\pythonstyle
\lstset{#1}
}
{}
% Python for inline
\newcommand\pythoninline[1]{{\pythonstyle\lstinline!#1!}}
% --------------------------------------------------------------------
    
% % list of acronyms
% \usepackage[acronym,nonumberlist,toc,numberedsection=autolabel]{glossaries}
% \iflanguage{czech}{\renewcommand*{\acronymname}{Seznam pou{\v z}it{\' y}ch zkratek}}{}
% \makeglossaries

\newcommand{\tg}{\mathop{\mathrm{tg}}} %cesky tangens
\newcommand{\cotg}{\mathop{\mathrm{cotg}}} %cesky cotangens

% % % % % % % % % % % % % % % % % % % % % % % % % % % % % % 
% ODTUD DAL VSE ZMENTE
% % % % % % % % % % % % % % % % % % % % % % % % % % % % % % 

\department{Katedra softwarového inženýrství}
\title{Systém pro skórování Ultimate~Frisbee~zápasů~-~backend}
\authorGN{Marek} %(křestní) jméno (jména) autora
\authorFN{Dostál} %příjmení autora
\authorWithDegrees{Marek Dostál} %jméno autora včetně současných akademických titulů
\supervisor{Ing. Jiří Hunka}
\acknowledgements{Doplňte, máte-li komu a za co děkovat. V~opačném případě úplně odstraňte tento příkaz.}
\abstractCS{V~několika větách shrňte obsah a přínos této práce v~češtině. Po přečtení abstraktu by se čtenář měl mít čtenář dost informací pro rozhodnutí, zda chce Vaši práci číst.}
\abstractEN{Sem doplňte ekvivalent abstraktu Vaší práce v~angličtině.}
\placeForDeclarationOfAuthenticity{V~Praze}
\declarationOfAuthenticityOption{4} %volba Prohlášení (číslo 1-6)
\keywordsCS{Nahraďte seznamem klíčových slov v češtině oddělených čárkou.}
\keywordsEN{Nahraďte seznamem klíčových slov v angličtině oddělených čárkou.}


\begin{document}

% TODO: Vyresit, jak budu formatovat odstavce.
% TODO: Zjistit, jak se vkladaji citace.

% \newacronym{CVUT}{{\v C}VUT}{{\v C}esk{\' e} vysok{\' e} u{\v c}en{\' i} technick{\' e} v Praze}
% \newacronym{FIT}{FIT}{Fakulta informa{\v c}n{\' i}ch technologi{\' i}}

\begin{introduction}
	%sem napište úvod Vaší práce
\end{introduction}

\chapter{Cíl práce}

Cílem mé bakalářské práce je vytvořit backend pro webovou aplikaci, která bude zajišťovat online
skórování zápasů a následné zobrazování statistik. Sou\-částí práce tak bude analýza dosavadních
i možných řešení. Výsledkem analýzy pak bude návrh a implementace webové aplikace. Frontend bude
vyvíjen v~rámci souběžné bakalářské práce Jaroslava Veselého. Mým úkolem je vytvořit backend,
jenž bude obsluhovatelný pomocí REST API. Vznikající systém má pracovní název Catcher.

%- Uvod do problematiky (Analyza problematiky)
%-- Co je to Ultimate Frisbee
%-- Uvod do aplikacnich rozhrani
%--- API
%
%- Pozadavky a zadani
%- Existujici reseni (Analyza dostupnych sluzeb)
%-- Catcher
%-- UltimateCentral
%
%- Požadavky a zadání
%- Analýza dostupných služeb
%  - Catcher
%  - UltimateCentral
%- Analýza dostupných technologií
%  - Výběr a detailnější popis vybraných technologií.
%- Návrh aplikace (db, model, atd.)
%  - Podivat se na to, co rikal Hunka.
%- Problemy pri implementaci
%- ...

\chapter{Úvod do problematiky}

\indent

V první kapitole bychom se měli seznámit s pojmem Ultimate Frisbee. Dále pak nastíníme
problematiku pořádání turnajů v tomto minoritním sportu.

\section{Co je to Ultimate Frisbee}

% TODO: Tady si mozna napsat neco uz hotovy na netu. Zaroven by bylo dobry zminit, jak se ultimate hraje.
% TODO: Obrazek???

\indent

Ultimate Frisbee je mladý a dynamicky se rozvíjející sport s létajícím talířem,
který se hraje od roku 1968 \cite{cald-o-ultimate}. Hraje jej přibližně sedm miliónů
hráčů ve více než 80 zemích světa a jeho popularita rok od roku stoupá \cite{usa-about-ultimate}.
Během posledních několika let je například běžné sledovat živé přenosy na sportovním kanálu ESPN
z Amerických soutěží, především profesionální ligy AUDL. Ultimate v roce 2015 dokonce získalo
uznání od Mezinárodního olympijského výboru \cite{cald-uznani}. Nejčastěji se hraje v kategoriích
open (muži), ženy, mix (smíšené týmy mužů a žen), junioři (do 19 let) a masters (nad 33 let).

\subsection{Pravidla}

Stručně popsaná pravidla podle Česká asociace lé\-ta\-jícícho disku:

\begin{quote}
``Ultimate je kolektivní bezkontaktní sport, v němž vítězí tým, který má na konci hrací doby
vyšší počet bodů. Hraje se na hřišti o rozměrech cca 100x37 metrů (délka fotbalového hřiště,
polovina jeho šířky). Na obou koncích hřiště jsou vyznačeny koncové zóny o hloubce cca 18 metrů.

V ultimate proti sobě hrají dva sedmičlenné týmy. Smyslem hry je pomocí přihrávek dopravit disk
do soupeřovy koncové zóny a jeho chycením v zóně získat bod. Po chycení disku se hráč musí
zastavit a do 10 vteřin disk přihrát spoluhráči. Povoleným pohybem hráče s diskem je pivotování,
tedy otáčení se kolem vlastní osy s jednou nohou pevně na zemi. V ultimate hráči často střídají
útok a obranu při ztrátě disku, ke které dochází záhozem disku do autu, na zem, jeho zachycením
soupeřem nebo při dlouhém držení disku. Není povolen fyzický kontakt mezi hráči ani přetahování
o disk.	''
\end{quote}

% TODO: Mam zminit, ze se hraje 5 na 5?

\subsection{Spirit of the Game}

\indent

Už od počátku je Ultimate Frisbee založeno na sportovním duchu, který klade odpovědnost
za fair play na samotné hráče. Předpokládá se vysoce soutěživá hra, ne však za cenu ztráty
vzájemné ohleduplnosti a vytracení radosti ze hry. Všechny přetupky na hřišti i mimo něj jsou
řešeny samotnými hráči. Jako jediná sportovní hra se tak obejde bez rozhodčích, a to i
na nejvyšších soutěžích, kterými jsou mistrovství Evropy a světa.

\medskip

Hraní fairplay je otázka cti. Na každém turnaji je vyhlašována cena Spirit of the Game,
která je cenou pro ty, kteří se chovali nejčestněji. Po každém zápase se týmy navzájem ohodnotí
v podobě číselného hodnocení a cenu pak získá tým s nejvyšším průměrem. Cena Spirit of the Game
je ceněna obdobně jako 1. místo.

\subsection{Ultimate v České republice}

\indent

V České republice zastřešuje sporty s létajícím talířem již od roku XXXX Česká asociace
lé\-ta\-jícícho disku (dále jen ``ČALD''). V celé republice eviduje XX zaregistrovaných klubů,
které se mezi sebou utkávají nejčastěji na víkendových turnajích. Těch bylo na našem
území za loňský rok více jak třicet, což je prakticky každý druhý týden v roce.

\medskip

Většina turnajů nebo mistrovství pak trvá zpravidla více dnů, během kterých se odehrají desítky
utkání. A s přibývajícím počtem hráčů a fanoušků vzniká čím dál větší poptávka po online
přenosech a statistikách z těchto akcí.

% TODO: Bude potreba doplnit, protoze se jedna o dulezity pojem, ktery bude pak zpracovan.

\subsection{Jak probíhá typický turnaj}

% TODO: dokoncit

\indent

Několik dní před turnajem se zveřejní rozpis, zpravidla v podobě dokumentu na Google Drive apod.
Ten je pak během turnajem editovaný a slouží jako jediný zdroj výsledků, který ... 

Nejdůležitejším údajem jsou výsledky z jednotlivých zápasů. Ty jsou nejčastěji zapisovány
na papír, který je vystaven na viditelném místě, aby si jej mohlo prohlédnout co nejvíce lidí.

\medskip

Částečně tyto procesy nahrazuje mobilní aplikace Catcher, ke které se ještě dostaneme.

% Vsechno se doposud pise na papir a je to proste cely na prd.
% TODO: Kdo vsechno, kolik lidi, na turnaji pobyva. Kolik probiha zapasu, treba i paralelne, kdo se o nej stara (navrh ma mobilni app). Co vsechno se da vycist z vysledku.
% TODO: Jeste doplnit dal.

\section{REST API}


\chapter{Specifikace požadavků}

Zadání vzniklo původně z požadavku na rozšíření mobilní aplikace Catcher, ke které se ještě
dostaneme v analýze již fungujících řešení. Se vznikajícími nároky na rozšiřitelnost přišel
také požadavek na RESTful rozhraní, které by backend obsluhovalo. V tu chvíli jsem tu byl
již já s nápadem na vytvoření nového backendu plně ovladatelným pomocí REST API.

\section{Požadavky na Catchera}

\indent

Na základě diskuse s ČALD, častými organizátory turnajů a některými hráči byly stanoveny
funkční a nefunkční požadavky na vznikající aplikaci.

\subsection{Funkční požadavky}

% TODO: Vice se rozepsat u jednotlivych bodu.


\begin{description}
  \item[Import a export dat] \hfill \\
  % TODO: Tohle neni jiste, zda uplne souhlasi.
  Systém umožňuje import oddílů a jeho hráčů z databáze ČALD.

  \item[Rozpis zápasů] \hfill 
  Oprávněná role může vytvořit svůj vlastní turnaj a vložit seznam všech zápasů,
  které se budou hrát. Vytvořený rozpis zápasů systém již doplňuje automaticky na základě
  odehraných výsledků (např. vítěze semifinále automaticky posune do finále). Ve chvíli,
  kdy to bude již možné, doplní celkové pořadí turnaje. Automatické doplnění se týká i 
  souhrné tabulky vzájemných hodnocení v kategorii SOTG.

  \item[Zadávání dat v rámci turnaje] \hfill \\
  Každý tým má možnost vytvořit vlastní soupisku na turnaj, kde bude hrát. Systém pak umožňuje
  v průbehu turnaje zadávat konkrétní údaje:
  \begin{itemize}
    \item průběžné skóre zápasů nebo jejich závěrečný výsledek
    \item skórující a asistující hráč
    \item hodnocení SOTG
  \end{itemize}
  
  \item[Tvorba statistik] \hfill \\
  Systém tvoří detailní statistiky hráčů, týmů a zápasů (obdržené a udělené body,
  počet asistencí, průměrná hodnota SOTG).
\end{description}

\subsection{Nefunkční požadavky}

% Zdroj: https://cs.wikipedia.org/wiki/Nefunk%C4%8Dn%C3%AD_po%C5%BEadavky_softwarov%C3%A9_architektury
% TODO: Muze se pridat Spravovatelnost (monitoring a dynamicka konfigurace aplikace)

\begin{description}
  \item[Snadná rozšířitelnost] \hfill \\
  Už nyní evidujeme změny, o které je zájem, ale nejsou předmětem této práce.
  I proto je nutné projekt dokončit tak, aby byl v budoucnu snadno rozšiřitelný nebo
  modifikovatelný.

  \item[Nízká cena] \hfill \\
  Cílem není vytvořit výdělečný projekt, ale fungující službu pro několik stovek hráčů
  a fanoušku v České republice. I proto je požadavkem použití volně dostupných
  knihoven a technologií.

  \item[Výkon] \hfill \\
  Podle celkem jednoduchých odhadů lze usoudit, že aplikaci budou čekat výkyvy v provozu.
  Většina zápisů a čtění dat probíhá během samotných akcí. Podle statistických údajů
  z již užívané aplikace Catcher víme, že během špičky není počet požadavků za sekundu
  větší než několik desítek. I proto není na celkový výkon kladen žádný zvláštní požadavek.
  Systém by každopádně měl více jak 95 \% žádostí zpracovat do 2 sekund (XXX milisekund) a na žádnou z nich
  nedpovědět za více jak 10 sekund (milisekund??).

  \item[Spolehlivost] \hfill \\
  Spolehlivost je základem pro funkční běh Catchera. Velká část operací, včetně chybných budou
  zapisovány do souborů pro snadné odhalení chyb. V případě výpadku by měl být informován
  administrátor pomocí SMS nebo emailu. Obnova musí být proveditelná ze zálohovacích souborů.
  
  % TODO: O vypadku nijak neinformuju, zalohu neprovadim
  
  \item[Bezpečnost] \hfill \\
  % autentizace a autorizace
  Systém musí jednoznačně určit a ověřit uživatele, který přistupuje k rozhraní Catchera. Zároveň
  musí existovat možnost za chodu přidávat, odebírat nebo měnit oprávnění. Dále pak systém musí
  být datově integritní, tzn. že obsah zpráv nebude při přenosu změněn.

  \item[Nároky na hardware] \hfill \\
  Systém musí být schopen běžet na běžných serverech s ne více jak 4GB RAM.

  \item[Formát importu] \hfill \\
  Pro import soupisek musí systém umět číst data ve formátu, který ČALD pro export používá.
  
  \item[Vytvoření dokumentace] \hfill \\
  Pro potřeby vývoje frontendu je potřeba vytvořit dostatečně detailní dokumentaci rozhraní API.
\end{description}

\section{Uživatelské role}

% TODO: Kdo vsechno bude do aplikace pristupovat
% TODO: Asi bych mel poznamenat do pozadavku, ze na kazdy turnaj je jine heslo.

\begin{description}
  \item[Nepřihlášený návštěvník] \hfill \\
  Jde o nejčastější přístup k aplikaci. Slouží k zobrazení všech statistik (aktuální skóre,
  statistika všech hráčů, hodnocení SOTG). Nemůže žádná data vytvářet nebo modifikovat
  a nevyžaduje přihlášení.
  
  \item[Organizátor] \hfill \\
  Účet pod touto rolí může vytvořit kdokoliv pouhou registrací. Po příhlášení lze přídávat
  nové turnaje a ty následně editovat. Tím se rozumí tvorba rozpisu zápasů a seznamu účastníků
  z již existujícího seznamu týmů. Dále může zadávat průběžné a výsledné skóre zápasů,
  skórující a asistující hráče a vidí na tabulku vzájemných hodnocení SOTG pro účely vyhlášení
  vítěze. Tato tabulka je pro ostatní role až do vyhlášení nezobrazitelná.
  
  \item[Klubový/oddílový účet] \hfill \\
  Pro účely odevzdávání hodnocení SOTG a úprav v týmové soupisce je vytvořen pro každý oddíl
  jeden společný účet. Tento účet má možnost vytvořit pouze admin. Přihlašovací údaje pak
  získá zástupce klubu a je pouze na něm, kdo z jeho oddílu bude spravovat vytvořený učet.
  
  \item[Admin] \hfill \\
  Má plnou kontrolu nad správou účtů a nad daty v databázi. Jeho možnosti zahrnují možnosti
  všech ostatních rolí.
\end{description}

\section{Případy užití}

Následující část popisuje nejběžnější případy užití. V návrhu RESTful rozhraní jde
o jednu z nejdůležitejších součastí, protože určuje směr návrhu. Výsledné API by tak mělo
pokrývat všechny níže uvedené případy.
% TODO: Mozna napsat neco o tom, ze kazde spravne api se dela tak, ze se zjisti vsechny usecasy, aby se mohli vytvorit dobre api.
% TODO: Co budou nejbeznejsi operace v aplikaci (zapisovani skore, zapisovani spiritu, tvorba turnaje a rozpisu). Tady to mozna napsat VSECHNO v bodech, protoze z toho budou vychazet API.

Následující seznam zachycuje jednotlivé požadavky a jejich popis.

% TODO: Napsat, kdo vsechno muzes tyto operace delat. Napr. organizator muze ukoncit zapas.

\begin{description}
 \item[F1: Import dat z databáze ČALD] \hfill \\
 Doplňuje seznam oddílů a jejich hráčů do databáze Catchera. V pravidelném intervalu stáhne
 aktuální data z ČALD databáze a doplní doposud aktuální data v databázi.

 % TODO: neexistuje zadny automaticky import
 
 \item[Získání libovolného oddílu z databáze] \hfill \\
 Získá množinu oddílů, jež jsou v databázi uloženy...
 % TODO: dokoncit
 
 \item[Získání libovolného hráče z databáze] \hfill \\
 ...
 % TODO: dokoncit
 
 \item[Vytvoření turnaje] \hfill \\
 Vytvoří konkrétní podobu turnaje. Do databáze uloží účastníky turnaje, základní skupiny,
 zápasy ve skupinách a následnující zápasy v play-off.
 
 \item[Získání libovolného turnaje z databáze] \hfill \\
 Získá množinu turnajů, jež jsou v databázi uloženy. S pomocí několika parametrů lze tuto
 množinu filtrovat:
 \begin{itemize}
    \item turnaj s konkrétním identifikátorem
    \item pouze ne/ukončené turnaje
    \item pouze ne/aktivní turnaje
    \item turnaje odehrávající se v čase od do
    \item turnaje v kategoriích open, women, mix apod.
    \item turnaje vytvořené konkrétním uživatelem
  \end{itemize}
  Prvek množiny obsahuje informace o turnaji, jako například název, popis, termín a místo turnaje.
 
 \item[Získá účastníky turnaje] \hfill \\
 Ke konkrétnímu turnaji získá kolekci všech týmů, jež se účastní. 
 
 \item[Získá týmovou soupisku] \hfill \\
 Ke konkrétnímu týmu získá kolekci všech hráčů, jenž se účastní.
 
 \item[Zadání skórujícího nebo asistujícího hráče] \hfill \\
 V průběhu zápasu umožňuje zadat skórujícího nebo asistujícího hráče.
 
 \item[Ukončit zápas s výsledným skóre] \hfill \\
 Ukončí zápas a uloží jeho výsledek do databáze.
 
 \item[Zadat hodnocení SOTG] \hfill \\
 Po ukončení zápasu lze z týmového účtu udělit soupeři hodnocení SOTG. Tento údaj se uloží do tabulky, kde jsou všechna vzájemná hodnocení. 
 
 \item[Zobrazit tabulku vzájemných hodnocení SOTG] \hfill \\
 Odešle uživateli vhodnou strukturu, kde budou jasně uvedeny všechna doposud vyplněná hodnocení kategorie SOTG.
 
 \item[Zobrazit pořadí týmů v kategorii SOTG] \hfill \\
 Vrátí pořadí týmů v kategorii SOTG, včetně průměru všech obdržených hodnocení, a informaci o tom, zda je toto pořadí již finální.
 
 \item[Ukončit turnaj] \hfill \\
 Změní stav turnaje na ukončený. Tabulka vzájemných hodnocení SOTG a výsledné pořadí se tím stanou veřejnými. 
 
 \item[Zobrazit výsledky všech zápasů] \hfill \\
 
 \item[Zobrazit detail konkrétního zápasu] \hfill \\ hráči, body, čas
 
 \item[Získá seřazené hráče podle bodování] \hfill \\ Seřadí hráče na základě bodovaní.
\end{description}

\chapter{Analýza}

\section{Existující řešení}

% TODO: Jake reseni uz existuji. Catcher, UltimateCentral atd., co je na nich dobryho a co ne

\section{Možnosti řešení}

% TODO: Co vsechno si z techto aplikaci muzu vzit. Pouzit API nebo jine technologie.

\section{Něco o API}

% TODO: Napsat neco o teorii REST API. Mozna lze sjednotit s predchozi kapitolou.

\chapter{Návrh}

\section{Návrh tříd a databáze}

% TODO: Takovy ty kecy teoreticky. Jak ma vypadat diagram trid, databazovy model apod.

\section{Architektura a návrh nasazení}

% TODO: Nevim, ale asi jak se to cely nasadi.

\chapter{Implementace}

\section{Výběr technologií}

\indent

V době, kdy existuje obrovské množství programovacích jazyků, jejich ekosystémů a dalších frameworků je důležité se umět nespálit. 
Vybrat špatné technologie totiž může znamenat prodloužení vývoje, zvýšení celkových nákladů nebo dokonce ukončení projektu.
Proto je nutnost tuto část dostatečně analyzovat a svědomitě zvážit, aby nedošlo k omylu, který by zabraňoval úspěšné dokončení.
Pro adekvátní výběr tak bylo potřeba specifikovat příslušná kritéria, které by jasně porovnala všechny zvažované technologie:

\subsubsection*{Podpora webových služeb}
Při tvorbě webové aplikace nemůže být ani řeč o technologii, která by pro použití na webu nebyla vhodná. Technologie musí
umět pracovat s mapováním URL a HTTP metod na zdroje a pracovat s běžnými formáty dat v internetové komunikaci (např. JSON).

\subsubsection*{Dokumentace a uživatelská podpora}
Pro pochopení technologie a její správné používání je často potřeba znát detaily, které se uživatel dozví z dokumentace.
Ta dokáže často výrazně zkrátit učící křivku a programátorovi tak šetří čas. Pokud dokumentace nestačí, může často její
nedostatky zachránit dostatečně široká uživatelská základna. Velká uživatelská podpora navíc snižuje riziko zestárnutí zvolené technologie.

\subsubsection*{Znalost technologie}
Pokud již programátor některou z technologií zná, může mu tato znalost ušetřit mnoho času při vývoji. Nemusí totiž novou technologii
nijak složitě zavádět nebo se učit její použití. Vhodným výběrem se také může vyvarovat situacím, kdy až v průběhu implementace narazí
na neřešitelné problémy. Zavrhnout neznámé technologie ale může znamenat přijít o možnost používat nástroje, které jsou problémům šité na míru. 

\subsubsection*{Výkon}
I když aplikace nepočítá s vetším provozem čítající stovky požadavků za sekundu, není důvod používat technologie, které jsou pomalé
nebo spotřebovávají příliš moc prostředků.

\section{Zvolené technologie}

\subsection{Programovací jazyk Python}

\indent

Python je víceúčelový skriptovací jazyk navrhnut v roce 1991 \cite{python-year} Guido van Rossumem.
Je vyvíjen jako open source projekt a je bezplatně dostupný pro většinu dostupných platforem (Unix, Windows, Mac OS).
V distribucích Linuxu je často součástí základní instalace. Díky velkému množství knihovních modulů z různých oblastí
má velice široké možnosti uplatnění. Běžně jej používají ve firmách jako Google, Dropbox, YouTube, Red Hat, Cisco,
Facebook nebo Microsoft \cite{python-companies}.

\medskip

Protože jde o dynamický interpretovaný jazyk, jeho zdrojový kód není nutné překládat překladačem do strojového kódu.
Je to zárovéň hybridní jazyk, protože umožňuje používat různá programovací paradigmata, včetně objektově orientovaného,
imperativního, procedurálního nebo v omezené míře i funkcionálního. I když je Python mnohokrát označován za skriptovací jazyk,
jeho návrh umožňuje psaní rozsáhlých a plnohodnotných aplikací včetně GUI. Podporuje také dynamickou kontrolu datových typů.

\medskip

Je to jazyk, který se velmi snadno učí a bývá považován za jeden z nejvhodnějších programovacích jazyků pro začátečníky.
Pomáhá tomu jeho jednoduchá syntaxe a čistota kódu. Na rozdíl od jiných jazyků bývá jeho zdrojový kód často krátký a dobře čitelný.
Je tak vhodný pro výuku i využití v praxi. Podle PYPL indexu\footnote{PopularitY of Programming Language Index je vytvořen analyzováním toho,
jak často jsou tutoriály jednotlivých programovacích jazyků hledány na Googlu.} je Python celosvětově nejrychleji roustoucí programovací jazyk
v~oblíbenosti za posledních pět let \cite{python-pypl}.
V dubnovém žebříčku roku 2016 drží druhé místo mezi jazyky Java a PHP. Mnoho dalších žebříčků pak uvádí Python vždy v prvních pěti místech.

\medskip

Po stránce výkonu je na tom Python relativně dobře, protože mnoho na~výkon náročných knihoven je implementováno v~jazyce C.
V porovnání s ostatními interpretovanými jazyky je na tom samotný Python taky dobře.
Fakt se kterým ale musíme počítat je ten, že dynamicky interpretované jazyky jsou obecně pomalejší, než kompilované jazyky.

\subsection{Webový framework Falcon}

% TODO: dovysvetlit, co je to UWSGI, mozna dat pred Falcon 

\indent

Falcon je minimalistický webový framework pro vývoj aplikačních backendů a jejich API s otevřeným zdrojovým kódem,
populární pro svoji neuvěřitelnou rychlost. Falcon ctí architektonický styl REST, což znamená, že mapuje použité
zdroje a jejich metody do HTTP protokolu.

\medskip

Falcon disponuje minimálním množstvím závislostí na jiných knihovnách a díky jeho flexibilitě je ho možné
používat ve většině verzích Pythonu\footnote{Vývoj v Pythonu poznamenalo v roce 2008 vydání nové verze Python 3.0,
která je částečně zpětně nekompatibilní.}. Kromě toho umí pracovat s WSGI. Nevýhodou Falconu je menší uživatelská
základna na rozdíl od konkurenčního Flasku, jenž patří mezi nejpopulárnější webové frameworky pro Python.

\begin{table}[htb]
 \centering
 \begin{tabular}{|l||c|c|c|c|}\hline
 \bfseries \bfseries framework & \bfseries req/sec & \bfseries $\mu$s/req & \bfseries výkon \\[2mm]
 \hline
 Falcon (0.3.0) & 21,858 & 46 & 8x \\
 \hline
 Bottle (0.12.8) & 12,583 & 79 & 4x \\
 \hline
 Werkzeug (0.10.4) & 4,708 & 212 & 2x \\
 \hline
 Pecan (0.8.3) & 3,442 & 291 & 1x \\
 \hline
 Flask (0.10.1) & 2,837 & 352 & 1x \\
 \hline
 \end{tabular}
 \caption{Výkonnostní test několika podobných webových frameworků pro~CPython~2.7.9 \cite{falcon-benchmarks}. Jde o~implementaci jazyka Python, kterou používá Catcher.}
\end{table}

\subsection{Databáze MySQL}

\indent

Relační datábáze, se kterou probíhá komunikace pomocí jazyka SQL, jak už název napovídá.
Díky svému výkonu, snadné použitelnosti (lze ji nainstalovat na Linux, Windows, OS X a další) a faktu,
že jde o volně šiřitelný software (je k dispozici i pod komerční licencí),
je MySQL velmi populární. Je součástí velmi oblíbené kombinace
základního softwaru na serverech známou pod zkratkou LAMP\footnote{Linux, Apache, MySQL, PHP}.

\medskip

Pro správu databáze se používá zpravidla příkazový řádek nebo lze separátně stáhnout
a nainstalovat nástroj zvaný MySQL Workbench. Ten byl použit i v této práci při návrhu databázového modelu.
Od svého vzniku v roce 1995 [ZDROJ] se od MySQL odpojilo několik alternativních větví (tzv. fork).
Mezi nejznámější případy patří MariaDB [ZDROJ] a Percona [ZDROJ].

\subsection{Peewee ORM}

\indent

Peewee je implementace ORM pro jazyk Python. Obsahuje podporu pro~databáze SQLite,
PostgreSQL a~MySQL. Je otevřeným softwarem a jeho zdrojový kód je dostupný na~GitHubu.

\subsubsection{Co je to ORM?}

\indent

ORM (\textit{Object-relational mapping}) je programovací technika, která zajišuje, že data z relační databáze
se automaticky konvertují do objektů v OOP. Programátor tak pracuje s perzistentními objekty,
místo psaní SQL dotazů. Pokročilejší ORM dokonce dokážou využít objektovou dědičnost, což relační databáze nepodporují.

\medskip

Technika je často kritizována, protože jde často o zbytečnou režii navíc
a programátory odnaučuje psát SQL dotazy, navíc efektivně.
Mnoho ORM nástrojů složitější dotazy přeloží do jazyka SQL tak neúčinně,
že dotaz může být mnohonásobně pomalejší, než kdyby je programátor napsal v SQL sám.
V závěru této práce jsem použití ORM zhodnotil.

\subsection{Verzovací systém Git}

\indent

Git je nástroj vytvořený Linusem Torvaldem (tvůrcem Linuxového jádra).
Slouží k distribuované správě verzí libovolných digitálních informací, například zdrojových kódů.
Výraznou výhodou je možnost spolupráce velkého množství programátorů na jednom softwarovém projektu. 
Každý programátor má adresář s projektem na svém lokálním disku a všechny změny sdílí s centrálním repozitářem,
ke kterému mají přístup i všichni ostatní programátoři. Ti si pak mohou stáhnout všechny změny,
které se na projektu staly. Git neuchovává úplný stav každé verze, ale pouze rozdíly mezi jednotlivými verzemi.
Tím výrazně šetří paměť.  

\medskip

Ruku v ruce jde s popularitou Gitu nahoru i služba GitHub.
Ta nabízí bezplatný i komerční hosting pro repozitáře softwarových projektů.
Zdarma, a díky tomu populární v dané oblasti, je GitHub pro open source projekty.
Funguje od roku 2008 a hostuje už více jak 11 miliónů repozitářů [ZDROJ].
Poskytuje mnoho dalších vlastností, jako například možnost diskutovat nad kódem
nebo zasílat notifikace o změnách.

\subsection{Webový server Nginx}

\indent

Nginx je webový server a reverzní proxy\footnote{Reverzní proxy se používá pro zvýšení výkonu webového serveru.
Rozděluje vstupující provoz na více serverů, například vyvažuje zátěž serverů zapojených v clusteru.}, který pracuje s běžnými protokoly.
První oficiální verze se objevila v roce 2004 \cite{nginx-changes}, kterou vyvynul Ruský softwarový inženýr Igor Sysoev.
Zaměřuje se na vysoký výkon a nízké nároky na pamět. Je používán velkými firmami
díky propracované možnosti rozložení zátěže. Z velkých firem je to například Netflix, Wordpress.com,
GitHub, Dropbox nebo Seznam.cz.

\medskip

Dnes je již s 25 \% druhým nejpoužívanějším webovým serverem v prvním milionu nejzatíženějších webů na světě.
Napříč celým webem je pak na třetím místě s přibližně 16 \%. I když je Apache HTTP Server stále jednička, 
jejich rozdíl se neustále zmenšuje \cite{nginx-statistic}. Nginx je open source.

% TODO: IDEA: Moznost prilozit obrazek s grafem nebo tabulkou.

\section{Adresářová struktura}

% TODO: tady se trochu vic rozepsat

\indent

Adresářová struktura popsaná v této sekci odpovídá struktuře, ve které projekt najdete na GitHubu.
Služba se dělí na dvě časti, v jedné se nachází aplikační logika a v druhé je jednoduchá webová stránka s dokumentací.

\medskip

Celou aplikační logiku najdeme v adresáři \texttt{/catcher}, kde se nachází zdroje a jejich mapování na URL,
mapování relační databáze na objekty nebo zdrojové kódy zajišťující autentizaci a autorizaci uživatelů.
Dokumentace se nachází v adresáři \texttt{/html}, kde jsou všechny důležité soubory pro použití JavaScriptu a~CSS na webu.

\medskip

Konfigurační soubory jsou uloženy v adresáři \texttt{/catcher/config}.
Nachází se zde přístupové údaje k databázím (produkční a testovací) nebo emailovému účtu pro zasílání nového hesla.
Konfigurační soubory nejsou uloženy v~repozitáři na Gitu. 

\begin{figure}[ht!]
\dirtree{%
  .1 /.\DTcomment{kořenová složka}.
  .2 /bin\DTcomment{spustitelné skripty}.
  .2 /catcher\DTcomment{aplikační logika}.
  .3 /api\DTcomment{vrstva spravující API}.
  .3 /config\DTcomment{konfigurační soubory}.
  .3 /logger\DTcomment{logování aplikace}.
  .3 /models\DTcomment{mapování relační databáze na OOP}.
  .3 /resources\DTcomment{zdroje}.
  .3 /test\DTcomment{testy}.
  .3 /restapi.py\DTcomment{spouštěcí skript}.
  .2 /html\DTcomment{adresář dostupný z webu (dokumentace)}.
  .2 /nginx\DTcomment{konfigurační soubory webového serveru}.
  .2 /sql\DTcomment{SQL skripty}.
  .2 /LICENSE\DTcomment{licence softwaru}.
  .2 /README.md\DTcomment{výčet závislostí a ostatní informace}.
  .2 /VERSION\DTcomment{aktuální verze}.
}
\caption{Struktura webové služby včetně adresáře pro webovou dokumentaci.\label{overflow}}
\end{figure}

\section{Bezpečnost}

\textit{Jak probíhá komunikace, jak se uživatelé autentizují a autorizují.}

Oauth

\section{Dokumentace}

Sphinx atd.

\section{Další}

\textit{Popíšu vybrané části systému. Například jak aplikace komunikuje s databází, nebo jak se počítají SOTG, postupy ze skupin apod. Nemám ještě dovymšlené.}

\chapter{Testování}

% IDEA: rozepsat se s rozdelenim dalsich testu

Jednou z nejdůležitejších součástí vývoje je testování. Pomáhá zkoumat funkčnost softwaru a v případě bezchybného
a úplného testování zajišťuje, že výsledná aplikace neobsahuje chyby a má požadované vlastnosti.
Testování se často provádí ve více iteracích, například pravidelně před vydáním nové verze softwaru.
Součástí testování je pak především reportování nalezených chyb.

Testy se dělí na tzv. black-box a white-box testy. Při testování černé skřínky nemáme informace o vnitřní podobě a porovnáváme
pouze správnost výstupních dat po zadání vstupních. Do testovaného subjektu se nelze podívat, vidíme jen to, jak se chová navenek.
Důvodem tohoto typu testování je analyzovat software z pohledu uživatele.
Na opačné straně se při znalosti implementace používá testování bílé skřínky. Sice tato metoda zastiňuje pohled uživatele,
ale tester je schopen lépe určit, kde hledat případné chyby. Při~částečné znalosti softwaru se ještě používá pojem testování šedé skřínky (gray-box).
Neznáme například přesnou podobu zdrojového kódu, ale jen použitý algoritmus použitý v aplikaci. 

Testy se také dělí na automatizované a manuální. Méně nákladnou a rychlejší cestou je zpravidla automatizované testování,
kdy se automaticky spouští velké množství testů, často i s velkým množstvím vstupních dat. Tyto testy se používají v místech,
kde dochází k opakování stejného nebo velmi podobného scénáře. Ne vždy se ale používá pouze automatizované testování.
Když test vyžaduje lidský úsudek, rozdílné přístupy nebo jej není nutné pravidelně opakovat, používá se manuální testování či kombinace obojího.

\section{Testování Catchera}

Testování softwaru je stále velmi podceňovaný obor a často je tato fáze vývoje odsunuta na druhou kolej nebo zcela vynechána.
Kromě toho, že psaní testů se ukázalo jako nejrychlejší způsob odhalování chyb, bylo pak ještě více překvapující, že při~použití
techniky TDD\footnote{Test-driven development - technika, kdy se testy píší ještě před samotným vývojem.} se znatelně zefektivnil
vývoj. Už~při~psaní zdrojového kódu jsem tak mohl jednoduše spouštět již vytvořené testy a znát jejich stav.
Díky specifickému popisu získání a~manipulace se zdroji v architektuře REST jsem většinu testů podřídil tomuto rozdělení a prováděl
jsem tzv. testování funkcionalit -- z pohledu uživatele, ověřování úzce zaměřených scénářů.

V Pythonu se nejrozšířenější framework pro automatizované testování jmenuje Unittest~\cite{python_unittest}. Já jsem však využil testovací nástroj, jenž přímo
poskytuje webový framework Falcon. Je potřeba zmínit, že tento nástroj většinu metod dědí od~základního Unittestu a~jeho možnosti pouze
rozšiřuje pro~lepší a~rychlejší testování REST API.

Automatizovanými testy jsou v~Catcheru pokryty všechny zdroje s~jejich metodami (celkem jich je 66). Typicky jsou testy po~několika sdružovány do~jedné třídy,
ve~které je možnost vytvořit specifické testovací prostředí. V knihovně \texttt{unittest} jsou pro tyto účely často používány metody \pythoninline{setUp()} a~\pythoninline{tearDown()},
které jsou volány před, respektive po každém testu. Důsledek tohoto chování je zajištění velmi důležitého aspektu v testování -- nezávislosti testů.
Bez něj by mohly testy pracovat nad stejnou množinou dat a navzájem si \uv{kazit} výsledky.
Nejčastější podoba testů pro~Catchera je vidět v~následující velmi zjednodušené ukázce kódu: 

\begin{python}
class TestClubs(falcon.testing.TestBase):

    def setUp(self):
      '''Spustí se před každým testem,
      zpravidla naplní testovací databázi daty.'''

    def tearDown(self):
      '''Spustí se po každém testu,
      zpravidla vyčistí testovací databázi.'''

    def testGet(self):
      '''Testuje požadavek GET - získá kolekci všech oddílů.'''     
      response = self.simulate_request('GET', '/api/clubs')
      self.assertEqual(self.srmock.status, HTTP_200)

    def testPost(self):
      '''Testuje požadavek POST - vytvoří nový oddíl.'''
      response = self.simulate_request('POST', '/api/clubs', body)
      self.assertEqual(self.srmock.status, HTTP_201)
\end{python}

Kromě porovnávání návratových HTTP kódů se v mým testech zkoumá podoba a obsah návratových dat, správné vyhazování vyjímek
nebo stav testovací databáze před a po volání metod.

% TODO: IDEA: Lze se zminit o testovani a prubezne integraci. Napriklad napsat, ze mame zajem v budoucnu pouzivat Codevoc nebo Travis CI.

\chapter{Nasazení}

% TODO: Popis jake technologie jsem si vybral na provoz aplikace na serveru: web server a uwsgi + uml diagram nasazeni

\section{Softwarové požadavky}

\section{Konfigurace}

\section{Údržba}

% \chapter{Analýza a návrh}
% \chapter{Realizace}

\begin{conclusion}
	%sem napište závěr Vaší práce
	% TODO: Budoucnost projektu.
\end{conclusion}

% TODO: Promyslet, jak presne odevzdavat citace.
%\bibliographystyle{csn690}
%\bibliographystyle{iso690}
%\bibliography{ref}

\begin{thebibliography}{1}

\bibitem{ultimate-time} BERNACCHI, Chris a Bryan WALSH. Don’t Let the IOC\
Ruin Ultimate Frisbee. TIME [online]. 2015 [cit. 2016-04-27].\
Dostupné z: http://time.com/3982671/dont-let-the-ioc-ruin-ultimate-frisbee/

\bibitem{cald-ultimate} Co je Ultimate? Česká Asociace Létjícího Disku [online].\
[cit. 2016-04-27]. Dostupné z: http://cald.cz/co-je-ultimate

\bibitem{usa-ultimate} What is Ultimate? USA Ultimate: About Ultimate [online].\
2015 [cit. 2016-04-27]. Dostupné z: http://www.usaultimate.org/about/

\bibitem{cald-uznani} KOTĚŠOVEC, Petr. ČALD získala oficiální uznání od Českého Olympijského Výboru.\
In: Česká Asociace Létjícího Disku [online]. 2015 [cit. 2016-04-27].\
Dostupné z: http://cald.cz/novinky/cald-ziskala-oficialni-uznani-od-ceskeho-olympijskeho-vyboru

\bibitem{cald-historie} Historie ultimate. Česká Asociace Létjícího Disku [online].\
[cit. 2016-04-27]. Dostupné z: http://cald.cz/historie-ultimate

\bibitem{cald-kalendar} Kalendař. Česká Asociace Létjícího Disku [online].\
[cit. 2016-04-27]. Dostupné z: http://cald.cz/kalendar

\bibitem{cald-catcher} VOSEČEK, Jiří, Ondřej BURKERT a Petr KOTĚŠOVEC.\
Mobilní aplikace pro skórování na turnajích.\
In: Česká Asociace Létjícího Disku [online]. 2013 [cit. 2016-04-29].\
Dostupné z: http://cald.cz/novinky/mobilni-aplikace-pro-skorovani-na-turnajich

\bibitem{catcher-play} Catcher. Google Play [online]. 2016 [cit. 2016-04-29].\
Dostupné z: https://play.google.com/store/apps/details?id=com.ulti.catcher\&hl=cs

\bibitem{http_metody} GOURLEY, David a Brian TOTTY. HTTP: the definitive guide.\
1st ed. Sebastopol, CA: O’Reilly, 2002, xviii, 635 p. ISBN 15-659-2509-2.

%GOURLEY, David a Brian TOTTY. HTTP: the definitive guide. 1st ed. Sebastopol, CA: O’Reilly, 2002, xviii, 635 p. ISBN 15-659-2509-2.

\bibitem{ultimate-organizer} Ultimate Organizer. SourceForge [online].\
2016 [cit. 2016-04-29]. Dostupné z: https://sourceforge.net/projects/ultiorganizer/

\bibitem{rest_vse} KOUDELKA, Jakub. Metodika návrhu REST API.\
Praha, 2013. Diplomová práce. Vysoká škola ekonomická v Praze. Vedoucí práce Lukáš Burkoň.

\bibitem{uri_wiki} URI scheme. In: Wikipedia: the free encyclopedia [online]. San Francisco (CA): Wikimedia\
Foundation, 2001- [cit. 2016-04-24]. Dostupné z: http://en.wikipedia.org/wiki/URI\_\
scheme

\bibitem{python-year} VAN ROSSUM, Guido. A Brief Timeline of Python.\
In: The History of Python [online]. 2009 [cit. 2016-04-26]. Dostupné z:\
http://python-history.blogspot.cz/2009/01/brief-timeline-of-python.html

\bibitem{python-companies} COCHRANE, Ken. Best Python Companies to Work For.\
In: DZone [online]. 2012 [cit. 2016-04-26]. Dostupné z:\
https://dzone.com/articles/best-python-companies-work

\bibitem{python-pypl} PYPL Index. PYPL PopularitY of Programming Language\
[online]. 2016 [cit. 2016-04-26]. Dostupné z: http://pypl.github.io/PYPL.html

\bibitem{falcon-benchmarks} Benchmarks. Falcon - The minimalist Python WSGI\
framework [online]. [cit. 2016-04-26]. Dostupné z: http://falconframework.org/

\bibitem{nginx-statistic} February 2016 Web Server Survey. Netcraft [online].\
2016 [cit. 2016-04-27]. Dostupné z: http://news.netcraft.com/archives/2016/02/22/february-2016-web-server-survey.html

\bibitem{nginx-changes} Changes. Nginx [online]. 2004 [cit. 2016-04-27].\
Dostupné z: http://nginx.org/en/CHANGES

\bibitem{fielding} FIELDING, Roy Thomas. Architectural styles and the design of network-based\
software architectures. Irvine, 2000. ISBN 0-599-87118-0. Disertační.\
University of California, Irvine. Vedoucí práce Richard Taylor.

\bibitem{uwsgi} PRAUS, Petr. Produkční nasazení Django aplikací na Cherokee pomocí WSGI.\
In: Zdroják.cz [online]. 2010 [cit. 2016-04-29]. Dostupné z:\
https://www.zdrojak.cz/clanky/produkcni-nasazeni-django-aplikaci-na-cherokee-pomoci-wsgi

\end{thebibliography}

\appendix

\chapter{Seznam použitých zkratek}
% \printglossaries
\begin{description}
	\item[GUI] Graphical user interface
	\item[XML] Extensible markup language
	\item[ČALD] Česká asociace létajícího disku
	\item[SOTG] Spirit of the Game
	\item[REST] Representational State Transfer
\end{description}

\chapter{Obsah přiloženého CD}

%upravte podle skutecnosti

\begin{figure}
	\dirtree{%
		.1 readme.txt\DTcomment{stručný popis obsahu CD}.
		.1 exe\DTcomment{adresář se spustitelnou formou implementace}.
		.1 src.
		.2 impl\DTcomment{zdrojové kódy implementace}.
		.2 thesis\DTcomment{zdrojová forma práce ve formátu \LaTeX{}}.
		.1 text\DTcomment{text práce}.
		.2 thesis.pdf\DTcomment{text práce ve formátu PDF}.
		.2 thesis.ps\DTcomment{text práce ve formátu PS}.
	}
\end{figure}

\end{document}

\iffalse
\fi