\begin{introduction}
 Létající talíř, tzv. frisbee, je pro mnoho lidí známým pojmem, ale o existenci kompetitivního kolektivního sportu,
 který se s tímto předmětem hraje už skoro 50 let, ví málokdo. Ultimate Frisbee se dnes hraje téměř
 ve všech koutech světa a jeho popularita v posledních letech raketově stoupá. Důsledkem je větší počet turnajů, na kterých se týmy mezi sebou utkávají.
 Organizace takové akce ale není vždy triviální záležitostí, proto je potřeba hledat možnosti, které pomohou k zefektivnění práce pořadatelů.
 
 V případě menších turnajů se často setkáváme s řešením, které je pro~jeden konkrétní případ dostačující.
 Výsledky zápasů se uchovávají například v~textových souborech, excelovských tabulkách nebo na statických webových stránkách.
 Pro konkrétnější využití dat nebo dlouhodobější sledování statistik jsou ale předchozí metody naprosto nevhodné.
 
 % TODO: hodne 'bude' v uvodu
 
 Hlavním cílem je navrhnout a implementovat backend pro webovou aplikaci, která bude zajišťovat správu turnajů, zaznamenávání
 zápasů a následné poskytování výsledků a statistik. Sou\-částí práce bude analýza dosavadních
 i~možných řešení, testování a nasazení na server. Backend, v mém případě webová služba bude bude obsluhovatelná
 pomocí veřejného rozhraní. Frontend bude vyvíjen v~rámci souběžné bakalářské práce Jaroslava Veselého.
 Vznikající systém bude mít pracovní název Catcher. Dílčími cíly je pak seznámení čtenáře s použitými technologiemi
 a sepsání dokumentace k API.
 
 Motivací při tvorbě práce mi byla především skutečnost, že sám jsem aktivním hráčem Ultimate Frisbee.
 Z toho důvodu disponuji velkou snahou a ctí celý projekt dotáhnout do úspešného konce.
\end{introduction}
 
 
  % TODO: zeptat se Hunky, zda nepletu jabka s hruskama